\documentclass{report}
\usepackage{float}

% for the image in the title
\usepackage{tikz}

% custom spacing
\usepackage{setspace}
\onehalfspacing

% footer and header
\usepackage{fancyhdr}
% \setlength{\headheight}{15.2pt}

% underlining
\usepackage{ulem}
\usepackage[T1]{fontenc}

% Table of contents link to corresponding sections
\usepackage{hyperref}
\hypersetup{
	colorlinks,
	citecolor=black,
	filecolor=black,
	linkcolor=black,
	urlcolor=black
}

\usepackage{amsmath}
% Remove che "Chapter" string before chapters
\iffalse
\makeatletter
\def\@makechapterhead#1{%
	\vspace*{50\p@}%
	{\parindent \z@ \raggedright \normalfont
		\interlinepenalty\@M
		\Huge\bfseries  \thechapter.\quad #1\par\nobreak
		\vskip 40\p@
}}
\makeatother
\fi

% Fancy chapters
\usepackage[Bjarne]{fncychap}
% options: Sonny, Lenny, Glenn, Conny, Rejne, Bjarne, Bjornstrup

\begin{document}


%title page
\begin{titlepage}
	\begin{figure}[t]
		\centering\includegraphics[width=0.3\textwidth]{images/unitn-logo}
	\end{figure}
	\begin{center}
		\textsc{ \LARGE{Università degli Studi di Trento \\}}
		\textsc{ \LARGE{Facoltà di Informatica\\ }}
		\textnormal{ \LARGE{Corso di Ingegneria del Software\\}}
		\vspace{30mm}
		\fontsize{10mm}{7mm}\selectfont
		\textup{Fix Mi \\ Application Implementation and Documentation}\\
	\end{center}

	\vspace{25mm}

	\centering
	\large Gruppo G43: \\ Giovanni Santini\\ Riginel Ungureanu \\ Valerio Asaro

	\vspace{20mm}

	\centering{\large{Anno Accademico 2023/2024 \\ Trento }}

\end{titlepage}




% use header and footers
\pagestyle{fancy}
\fancyhead[R]{\chaptername\ \thechapter}  % header

%\maketitle
\tableofcontents
\newpage



\section{Scopo del documento}


\section{Implementazione}

\section{Informazioni del Documento}

% table
\begin{center} % center the table
	\centering
	\begin{tabular}{ |p{4cm}|p{4cm}|  }
		\hline
		\centering Campo        & \qquad\qquad Valore                          \\ % I found no other way...
		\hline
		Titolo del Documento    & Application Implementation and Documentation \\
		\hline
		Titolo del Progetto     & Fix Mi                                       \\
		\hline
		Autori del Documento    &
		Giovanni Santini                                                       \\ & Riginel Ungureanu \\ & Valerio Asaro \\
		\hline
		Amministratore Progetto & Riginel Ungureanu                            \\
		\hline
		Versione del documento  & 1.0                                          \\
		\hline
	\end{tabular}
\end{center}



\chapter{Backend}

Il seguente capitolo descrive l'implementazione, con il supporto di diagrammi "UML" e codice, della parte "Backend" dell'applicazione, in riferimento a quanto detto nei precedenti documenti quali "Analisi dei Requisiti", "Specifica dei Requisiti" e del "Documento di Architettura".  Il capitolo è strutturato partendo da una visione generale dell'infrastruttura, andando verso una visione più specifica e dettagliata.


Gli estratti di codice verranno mostrati nel seguente formato:
\begin{figure}[H]
	\centering\includegraphics[width=1\textwidth]{images/example_code_01.png}
	Esempio di estratto di codice
\end{figure}

\subsection*{Tecnologie}

In questo capitolo verranno menzionate più volte alcune tecnologie, dunque ne viene riportata una breve descrizionee sotto:
\begin{itemize}
	\item "Docker": tecnologia che raccoglie il software in unità standardizzate chiamate container, offrendo tutto il necessario per la loro corretta esecuzione, incluse librerie, strumenti di sistema, codice e runtime. Il logo ricorda una balena di colore azzurro con dei rettangoli al di sopra rappresentanti dei containers, come se la balena stessa fosse una nave.
	\item "Docker Compose": software per la gestione di molteplici docker containers contemporaneamente. L'icona rappresenta un polpo con dei parallelepipedi azzurri tra i tentacoli.
	\item "Nginx": intermediario tra le richieste da parte dei client e il server. Può essere usato come load balancer, cache o proxy. Il logo è raffigurato da un N bianca con sfondo verde.
	\item "MongoDB": DBMS non relazionale. Il logo rappresenta una foglia verde.
	\item "Express": framework backend per applicazioni web in javascript. Il logo è formato dalle due lettere "e" e "x" in nero.
	\item "React": framework frontend lo sviluppo di applicazioni web. L'icona ricorda un atomo.
\end{itemize}

\section{Architettura dei Microservizi}

L'architettura proposta si basa sulla divisione logica (e fisica) delle funzionalità dell'applicazione tramite la distinzione di \textit{Microservizi}. Ciascun microservizio è indipendente dagli altri ed è composto a sua volta da una frontend e una backend distinte. Per tale ragione, è più accurato parlare di micro-frontend e micro-backend.

\subsection*{Vantaggi di un'architettura basata su microservizi}

Da un punto di vista di sviluppo, un'architettura non monolitica permette lo sviluppo asincrono dei singoli microservizi, oltre a facilitare la divisione del lavoro nelle varie parti. Tale architettura ha anche dei vantaggi a livello di performance in quanto il carico di lavoro chel'applicazione processa viene distribuito su più processi diminuendo il carico sul singolo. Altri vantaggi sono la scalarità dia una componente dell'applicazione in base alle esigenze e la ridondanza in quanto possono esistere più istanze di un microservizio contemporaneamente.

\subsection*{Svantaggi}

Un'architettura a microservizi è intrinsecamente più complessa nella progettazzione e nella menutenzione rispetto ad una architettura tradizionale. La necessità di orchestrare e mettere in comunicazione i diversi microservizi richiede particolari accortezze nella parte di design e di deploy. Tali problematiche sono state valutate con cura dal team di sviluppo.

\section{Visione Generale}
Prima di analizzare il singolo microservizio, questa sezione illustra una visione generale dell'applicazione e degli strumenti utilizzati per la realizzazione dell'infrastruttura, per poi concentrarsi sulle parti comuni di ogni microservizio e infine sul singolo microservizio.


\subsection*{Infrastruttura}

Si presti attenzione alla seguente infografica dell'infrastruttura implementata:

\begin{figure}[H]
	\centering\includegraphics[width=1\textwidth]{images/diagramma_microservizi.png}
	Schema dei microservizi
\end{figure}

L'intera infrastruttura viene inizializzata con \textit{Docker Compose}. In particolare, \textit{Docker Compose} si occupa di impostare i containers sullo stesso network con un IP statico e le variabili di ambiente come le porte e gli IP dei rispettivi microservizi, oltre ad inizializzare i containers, le porte e volumi condivisi.

Se eseguito su un singolo host, il network di default segue la seguente struttura:
\begin{figure}[H]
	\centering\includegraphics[width=1\textwidth]{images/network.png}
	Schema del network dei microservizi
\end{figure}

Docker imposta i microservizi sul network 10.5.0.0/24 con i seguenti IP:
% table
\begin{center} % center the table
	\centering
	\begin{tabular}{ |p{4cm}|p{4cm}|  }
		\hline
		\centering Nome Microservizio     & \qquad\quad Network IP \\ % I found no other way...
		\hline
		MongoDB                           & 10.5.0.10/24           \\
		\hline
		Microservizio Autenticazione      & 10.5.0.11/24           \\
		\hline
		Microservizio Task                & 10.5.0.12/24           \\
		\hline
		Microservizio Home                & 10.5.0.13/24           \\
		\hline
		Microservizio Gestione Dipendenti & 10.5.0.14/24           \\
		\hline
		Reverse Proxy                     & 127.0.0.1              \\
		\hline
	\end{tabular}
\end{center}

Ad ogni servizio sono state assegnate le seguenti porte:

\begin{center} % center the table
	\centering
	\begin{tabular}{ |p{4cm}|p{4cm}|  }
		\hline
		\centering Nome servizio & \qquad\qquad Porta \\ % I found no other way...
		\hline
		MongoDB                  & 27017              \\
		\hline
		Backend                  & 3001               \\
		\hline
		frontend                 & 3002               \\
		\hline
		Reverse Proxy            & 7777               \\
		\hline
	\end{tabular}
\end{center}

Due microservizi particolari sono il database e il reverse proxy. Il database fornisce la possibilità di salvare in modo permanente i dati dell'applicazione, mentre il reverse proxy permette di accedere facilmente all'ip di un microservizio attraverso una mappatura di ip. I software scelti sono rispettivamente \textit{MongoDB} e \textit{Nginx}.

In particolare, \textit{Nginx} associa le routes nel seguente modo:
\begin{figure}[H]
	\centering\includegraphics[width=1\textwidth]{images/nginx.png}
	Schema delle routes di Nginx
\end{figure}

\subsection*{Codice: Struttura}

La directory radice del progetto contiene i seguenti files:

\begin{figure}[H]
	\centering\includegraphics[width=1\textwidth]{images/tree.png}
	Output del comando "tree", mostra i contenuti della directory root del progetto.
\end{figure}

Segue una descrizione degli stessi:
\begin{itemize}
	\item "db": volume condiviso tra il docker container del database e il filesystem dell'host per mantenere persistenza dei dati quando il container viene riavviato o rimosso.
	\item "docker-compose.yaml": file di configurazione utilizzato da docker compose per inizializzare l'infrastruttura. Contiene le informazioni per avviare gli altri microservizi.
	\item "fixmi-microservice-gestione-dipendenti": cartella contenente il microservizio "Gestione Dipendenti".
	\item "fixmi-microservizio-autenticazione": cartella contenente il microservizio "Autenticazione".
	\item "fixmi-microservizio-home": cartella contenente il microservizio "Home".
	\item "fixmi-microservizio-task": cartella contenente il microservizio "task".
	\item "fixmi-reverse-proxy": cartella contenente il reverse proxy.
	\item "init\_scripts": script per inizializzare il database con dei dati di esempio.
	\item "README.md": contiene informazioni e documentazione sul deploy dell'infrastruttura.
\end{itemize}

\subsection*{Codice: docker-compose.yaml}

Si prenda come esempio questo frammento di codice responsabile del setup del microservizio task:
\begin{figure}[H]
	\centering\includegraphics[width=1\textwidth]{images/docker_code_01.png}
	Estratto dal file "docker-compose.yaml"
\end{figure}

In questo codice, notiamo che le porte vengono impostate sotto la sezione "ports", così come i volumi e il network. Le variabili d'ambiente contenenti i vari ip e le credenziali del database sono contenute nel file ".env" presente nella radice della cartella.
La configurazione per gli altri microservizi è simile, con qualche piccola differenza. Qua sotto vengono riportati gli altri microservizi:
\begin{figure}[H]
	\centering\includegraphics[width=1\textwidth]{images/yaml_gestione_dipendenti.png}
	Codice responsabile per l'avvio del microservizio "Gestione Dipendenti"
\end{figure}
\begin{figure}[H]
	\centering\includegraphics[width=1\textwidth]{images/yaml_autenticazione.png}
	Codice responsabile per l'avvio del microservizio "Autenticazione"
\end{figure}
\begin{figure}[H]
	\centering\includegraphics[width=1\textwidth]{images/yaml_home.png}
	Codice responsabile per l'avvio del microservizio "Home"
\end{figure}
\begin{figure}[H]
	\centering\includegraphics[width=1\textwidth]{images/yaml_database.png}
	Codice responsabile per l'avvio del database
\end{figure}
\begin{figure}[H]
	\centering\includegraphics[width=1\textwidth]{images/yaml_reverse_proxy.png}
	Codice responsabile per l'avvio del reverse proxy
\end{figure}

\subsection*{Codice: nginx.conf}

Il file \textit{nginx.conf} all'interno della cartella \textit{fixmi-reverse-proxy} imposta la configurazione per il reverse proxy.


TODO

\subsection*{Codice: init-mongo.js}

Il file contiene l'inizializzazione delle entrate del database in formato json. Sono stati individuati due database: "Tasks" e "Users" e le rispettive collezioni "tasks" e "users" (da notare la differenza della prima lettera da maiuscola a minuscola). I contenuti sono i seguenti:
\begin{figure}[H]
	\centering\includegraphics[width=1\textwidth]{images/db-tasks.png}
	Inizializzazione database "Task"
\end{figure}

\begin{figure}[H]
	\centering\includegraphics[width=1\textwidth]{images/db-users.png}
	Inizializzazione database "Users"
\end{figure}


\section{Parti comuni ad ogni microservizio}

In questa sezione verrà descritta la struttura che accomuna i vari microservizi.

\subsection{Scelta del Linguaggio}

Il linguaggio scelto per la backend e frontend è \textit{Typescript}: un linguaggio tipizzato che viene tradotto da un compilatore ("transpiler") in Javascript e dunque può essere eseguito in un web browser. Qualsiasi script JavaScript è anche codice TypeScript valido, questo permette all'applicazione di utilizzare l'ampia collezione di pacchetti JavaScript.

\subsection{Frameworks e Dipendenze}

Ogni microservizio utilizza:
\begin{itemize}
	\item \textit{ts-node} v10.9.2: ambiente runtime Typescript open source e multipiattaforma per la creazione del server di backend.
	\item \textit{eslint} v8.56.0: analizzatore di codice statico.
	\item \textit{nodemon} v3.0.3: tool di assistenza per Node.js. permette il riavvio automatico del server quando viene rilevata una modifica nel codice.
	\item \textit{express} v4.18.2: framework web per Node.js, utilizzato per la realizzazione di api.
	\item \textit{React} v18.3.0: libreria UI per javascript, utilizzata per la frontend.
	\item \textit{tailwindcss} v3.4.1: framework CSS.
	\item \textit{Jest} v29.7.0 : framework per il testing
	\item \textit{supertest} v7.0.0: framework per il testing di api, utilizzato insieme a Jest.
	\item \textit{concurrently} v8.2.2: esegue più comandi contemporaneamente, utilizzato per eseguire frontend e backend insieme.
	\item \textit{serve} v14.2.3: hosta un sito statico, utilizzato per hostare la frontend in produzione.
	\item \textit{dotenv} v16.4.5: permette l'accesso alle variabili di ambiente, utilizzate per la configurazione dei microservizi.
	\item \textit{body-parser} v1.20.2: permette di leggere il body di una richiesta http.
	\item \textit{mongodb} v6.5.0: interfaccia per la connessione al microservizio database.
\end{itemize}

\subsection{Struttura}

La cartella di un microservizio presenta i seguenti files:
\begin{figure}[H]
	\centering\includegraphics[width=1\textwidth]{images/structure.png}
	Output del comando "tree" all'interno di un microservizio.
\end{figure}
Segue una descrizione degli stessi, dall'alto verso il basso:
\begin{itemize}
	\item "backend": cartella contenente tutto il codice di Backend in TypeScript.
	\item "build": cartella contenente l'applicazione web statica generata da React.
	\item "commands": cartella contenente degli appunti degli sviluppatori durante la creazione del progetto.
	\item "dist": cartella contenente il codice backend per la produzione, generato dal compilatore di TypeScript.
	\item "Dockerfile": file contenente le istruzioni per la generazione del contenitore di docker.
	\item "nix-shell.nix": file contenente l'ambiente di sviluppo per chi sviluppa su NixOS.
	\item "node\_modules/": cartella contenente i moduli utilizzati dall'applicazione
	\item "package.json": file che definise le dipendenze e le impostazioni di vari moduli.
	\item "package-lock.json": contiene le versioni delle dipendenze dei moduli.
	\item "postcss.config.js": script di configurazione necessario per tailwindcss
	\item "public": contiene file di accesso pubblico per la frontned, come favicon.ico, robots.txt, index.html.
	\item "README.md": contiene le informazioni del progetto e le istruzioni per il deploy.
	\item "src/": cartella contenente tutto il codice per la frontend in tsx.
	\item "tailwind.config.js": file di configurazione per il funzionamento di tailwindcss
	\item "tsconfig.json": configurazione per TypeScript.
\end{itemize}

\subsection{Comandi npm}
Vengono definiti i seguenti scripts per l'esecuzione dell'applicazione tramite npm. Ogni comando può esssere eseguito digitando "npm run" ed il nome del comando.
\begin{itemize}
	\item "startfront": esegue solo la frontend, si aggiorna automaticamente ad ogni modifica del codice
	\item "buildfront": genera la frontend statica per la production
	\item "startback": esegue soltanto il server di backend, si aggiorna automaticamente ad ogni modifica del codice
	\item "start": esegue sia la frontend sia la backend in developement mode
	\item "production": esegue sia la frontend sia la backend in production mode
	\item "lint": esegue un'analisi stica del codice
	\item "test": esegue gli script di testing del codice.
\end{itemize}


\section{Microservizio Autenticazione}

Questa sezione comprende lo sviluppo e la documentazione del backend del Microservizio Autenticazione
\subsection*{Struttura del backend}
La struttura del backend del microservizio autenticazione è riportata
nella figura sotto.
\begin{figure}[H]
	\centering\includegraphics[width=0.5\textwidth]{images/microservizio-autenticazione/backend-structure.png}
	\caption{Output del comando "tree" all'interno di un microservizio.}
\end{figure}

\subsection*{Modellazione dati nel database}
\subsection*{Specifica delle risorse}
Di seguito le risorse estratte dal diagramma delle classi e utilizzate da questo microservizio.
\paragraph*{<<resource>> Cliente}
Rappresenta il profilo di un Cliente.
Ha i seguenti attributi
\begin{itemize}
	\item id
	\item email
	\item password\_hash
	\item permissionLevel
	\item token
\end{itemize}
\paragraph*{<<resource>> Dipendente}
Rappresenta il profilo di un Dipendente.
Ha tutti gli attributi di cliente, in più:
\begin{itemize}
	\item nome
	\item cognome
	\item data di assunzione
	\item data di nascita
	\item worktags
\end{itemize}
\paragraph*{Manager}
Rappresenta il profilo del Manager.
Ha tutti gli attributi di Dipendente, ma ha come permissionLevel "Manager"

Metodi
\paragraph{<<resource>> login}
Permette a un utente già esistente di accedere all'applicazione.
I parametri in ingresso sono email, password e codice 2FA.
E' un'API del backend.

\paragraph*{<<resource>> logout}
Permette a un utente che ha già eseguito l'accesso di svolgere il logout dall'applicazione.
Richiede in ingresso il token di sessione ricevuto al login.

\paragraph*{<<resource>> twofa}
Permette a un utente che vuole eseguire l'accesso, registrarsi per la prima volta, cambiare la password o rimuovere il proprio account di ricevere attraverso email un codice 2FA.
Richiede in ingresso la propria email.

\paragraph*{<<resource>> registrazione}
Permette a un nuovo cliente di registrarsi all'applicazione.
Richiede in ingresso l'email, la password e il codice 2FA

\paragraph*{<<resource>> cambioPassword}
Permette a un utente già esistente  di cambiare la propria password.
Richiede in ingresso l'email, la nuova password e il codice 2FA.

\paragraph*{<<resource>> eliminaProfilo}
Permette a un utente già esistende di eliminare il proprio profilo.
Richiede in ingresso l'email, la password e il codice 2FA.

\paragraph*{<<resource>> autenticaUtente}
Permette a un altro microservizio di autenticare un utente che desidera utilizzare una funzionalità che richiede autorizzazione.
Richiede in ingresso il token dell'utente.

\subsection{Login}
\subsubsection*{Specifica}
Questa API, all'indirizzo /api/auth/login, ha un metodo POST e viene utilizzata per far accedere utenti all'applicazione.
La request body è in formato x-www-form-urlencoded e deve contenere i seguenti campi:
\begin{itemize}
	\item email
	\item password
	\item twofa
\end{itemize}
La response body è in formato application/json. Di seguito le possibili risposte:
% table
\begin{center} % center the table
	\centering
	\begin{tabular}{ |p{4cm}|p{5cm}|p{4cm}| }
		\hline
		\centering Status Code & \qquad\quad Body e Cookie                               & \qquad\qquad Spiegazione                              \\ % I found no other way...
		\hline
		200 OK                 & \{text: "successfully logged in", token\} cookie: token & Login avvenuto con successo                           \\
		\hline
		400 BAD REQUEST        & \{error: "missing fields", missingFields\}              & Alcuni campi non sono stati specificati nella request \\ % I found no other way...
		\hline
		404 NOT FOUND          & \{error: "user not found"\}                             & non esiste utente con la mail fornita                 \\% I found no other way...
		\hline
		404 NOT FOUND          & \{error: "wrong password" \}                            & la password inserita non è corretta                   \\
		\hline
		400 BAD REQUEST        & \{error: "2fa not correct" \}                           & Il codice twofa inserito non è corretto.              \\
		\hline
	\end{tabular}
\end{center}
\begin{figure}[H]
	\centering\includegraphics[width=1\textwidth]{images/microservizio-autenticazione/diagrams/diagramma_login.drawio.png}
	\caption{Diagramma dell'endpoint "login"}
\end{figure}


\subsubsection*{Sviluppo}
Questa API viene utilizzata da un utente per eseguire l'accesso.
Una volta ricevuta la richiesta, viene controllata la presenza di tutti i campi richiesti.
Successivamente si controlla se l'utente esista o meno nel database, e successivamente la validità della password.
Dopo aver controllato che il codice 2FA sia corretto, viene generato un token, aggiunto alla sessione e inviato sia nel body sia come cookie nella risposta.


\begin{figure}[H]
	\centering\includegraphics[width=1\textwidth]{images/microservizio-autenticazione/login-carbon.png}
	Codice dell'API api/auth/login
\end{figure}
\subsubsection*{Documentazione}
Di seguito la specifica openapi dell'endpoint login contenuta nel file 'openapi.yaml' della directory del microservizio
\begin{verbatim}
	/login:
    post:
      summary: Login method for user authentication
      description: This endpoint allows users to log in by providing email, password, and two-factor authentication code.
      requestBody:
        required: true
        content:
          application/x-www-form-urlencoded:
            schema:
              type: object
              properties:
                email:
                  type: string
                  format: email

                password:
                  type: string
                  format: password
                  
                twofa:
                  type: string
                  format: twofa-code
               
              required:
                - email
                - password
                - twofa
      responses:
        '200':
          description: Successfully logged in
          content:
            application/json:
              schema:
                type: object
                properties:
                  text:
                    type: string
                    example: "successfully logged in!"
                  token:
                    type: string
                    example: "12415343463452"
            

          headers:
            Set-Cookie:
              schema:
                type: string
              description: Session token
        '400':
          description: Bad Request - Missing fields or 2fa not correct
          content:
            application/json:
              schema:
                oneOf:
                  - $ref: '#/components/schemas/missingFieldsSchema'
                  - $ref: '#/components/schemas/wrongTwoFaSchema'
      
        '404':
          description: User not found or wrong password
          content:
            application/json:
              schema:
                oneOf:
                  - $ref: '#/components/schemas/notFoundSchema'
                  - $ref: '#/components/schemas/wrongPassSchema'
\end{verbatim}
\subsubsection*{Testing}
Per questo endpoint sono stati creati i seguenti test:
\begin{figure}[H]
	\centering\includegraphics[width=1\textwidth]{images/microservizio-autenticazione/tests/login_test_1.png}
	\caption{Test Login 1}
\end{figure}
\begin{figure}[H]
	\centering\includegraphics[width=1\textwidth]{images/microservizio-autenticazione/tests/login_test_2.png}
	\caption{Test Login 2}
\end{figure}
\begin{figure}[H]
	\centering\includegraphics[width=1\textwidth]{images/microservizio-autenticazione/tests/login_test_3.png}
	\caption{Test Login 3}
\end{figure}
\begin{figure}[H]
	\centering\includegraphics[width=1\textwidth]{images/microservizio-autenticazione/tests/login_test_4.png}
	\caption{Test Login 4}
\end{figure}
\subsection{Logout}
\subsubsection*{Specifica}


Questa API, all'indirizzo/api/auth/logout, ha un metodo DELETE e viene utilizzata da un utente autenticato per svolgere il logout.
La request body è in formato x-www-form-urlencoded e deve contenere il token dell'utente.
La response body è in formato application/json. di seguito le possibili risposte:
% table
\begin{center} % center the table
	\centering
	\begin{tabular}{ |p{4cm}|p{5cm}|p{4cm}| }
		\hline
		\centering Status Code & \qquad\quad Body e Cookie                        & \qquad\qquad Spiegazione                               \\ % I found no other way...
		\hline
		200 OK                 & \{text: "successfully logged out"\}              & Logout avvenuto con successo                           \\
		\hline
		400 BAD REQUEST        & \{error: "missing fields", missingFields\}       & Alcuni campi non sono stati specificati nella request  \\ % I found no other way...
		\hline
		404 NOT FOUND          & \{error: "user not found with the given token"\} & non esiste utente nella sessione con il token fornito. \\% I found no other way...
		\hline
		404 NOT FOUND          & \{error: "wrong password" \}                     & la password inserita non è corretta                    \\
		\hline
	\end{tabular}
\end{center}
\begin{figure}[H]
	\centering\includegraphics[width=1\textwidth]{images/microservizio-autenticazione/diagrams/diagramma_logout.drawio.png}
	\caption{Diagramma dell'endpoint "logout"}
\end{figure}
\subsubsection*{Sviluppo}
Questa API viene utilizzata da un utente già autenticato per eseguire il logout.
Una volta ricevuta la richiesta, viene controllata la presenza di tutti i campi richiesti.
Successivamente si controlla se il token fornito è presente nella sessione.
A operazione completata, viene rimosso il token dalla sessione.


\begin{figure}[H]
	\centering\includegraphics[width=1\textwidth]{images/microservizio-autenticazione/logout-carbon.png}
	Codice dell'API api/auth/logout
\end{figure}
\subsubsection*{Documentazione}
Di seguito la specifica openapi dell'endpoint logout contenuta nel file 'openapi.yaml' della directory del microservizio
\begin{verbatim}
	/logout:
    delete:
      summary: Method for logging out 
      description: This endpoint allows users who have already logged in to log out by providing their token.
      requestBody:
        required: true
        content:
          application/x-www-form-urlencoded:
            schema:
              type: object
              properties:
                token:
                  type: string
              required:
                  - token
      responses:
        '200':
          description: Succesfully logged out
          content:
            application/json:
              schema:
                type: object
                properties:
                  text:
                    type: string
                    example: "Succesfully logged out!"
        '400':
          description: Bad Request - Missing fields 
          content:
            application/json:
              schema:
                $ref: '#/components/schemas/missingFieldsSchema'
        '404':
          description: User not found with the given token
          content:
            application/json:
              schema:
                $ref: '#/components/schemas/notFoundSchema'
	
\end{verbatim}
\subsubsection*{Testing}
\begin{figure}[H]
	\centering\includegraphics[width=1\textwidth]{images/microservizio-autenticazione/tests/logout_test_1.png}
	\caption{Test Logout 1}
\end{figure}
\begin{figure}[H]
	\centering\includegraphics[width=1\textwidth]{images/microservizio-autenticazione/tests/logout_test_2.png}
	\caption{Test Logout 2}
\end{figure}
\begin{figure}[H]
	\centering\includegraphics[width=1\textwidth]{images/microservizio-autenticazione/tests/logout_test_3.png}
	\caption{Test Logout 3}
\end{figure}
\subsection{Registrazione}
\subsubsection*{Specifica}
Questa API, all'indirizzo /api/auth/register, ha un metodo POST e viene utilizzata per la creazione di un account cliente.
La request body è in formato x-www-form-urlencoded e deve contenere i seguenti campi:
\begin{itemize}
	\item email
	\item password
	\item twofa
\end{itemize}
La response body è in formato application/json. Di seguito le possibili risposte:
% table
\begin{center} % center the table
	\centering
	\begin{tabular}{ |p{4cm}|p{5cm}|p{4cm}| }
		\hline
		\centering Status Code & \qquad\quad Body e Cookie                                & \qquad\qquad Spiegazione                              \\ % I found no other way...
		\hline
		200 OK                 & \{text: "successfully registered", token\} cookie: token & Login avvenuto con successo                           \\
		\hline
		400 BAD REQUEST        & \{error: "missing fields", missingFields\}               & Alcuni campi non sono stati specificati nella request \\ % I found no other way...
		\hline
		404 NOT FOUND          & \{error: "user already exists with the given email"\}    & Esiste già un utente con la mail fornita              \\% I found no other way...
		\hline
		400 BAD REQUEST        & \{error: "2fa not correct" \}                            & Il codice twofa inserito non è corretto.              \\
		\hline
	\end{tabular}
\end{center}
\begin{figure}[H]
	\centering\includegraphics[width=1\textwidth]{images/microservizio-autenticazione/diagrams/diagramma_register.drawio.png}
	\caption{Diagramma dell'endpoint "register"}
\end{figure}
\subsubsection*{Sviluppo}
Questa API viene utilizzata da un utente per creare un proprio account cliente.
Una volta ricevuta la richiesta, viene controllata la presenza di tutti i campi richiesti.
Successivamente si controlla se un utente con la stessa email sia già esistente nel database.
Dopo aver controllato che il codice 2FA sia corretto,viene inserito il nuovo profilo cliente nel database, viene generato un token, aggiunto alla sessione e inviato sia nel body sia come cookie nella risposta.


\begin{figure}[H]
	\centering\includegraphics[width=1\textwidth]{images/microservizio-autenticazione/register-carbon.png}
	Codice dell'API api/auth/register
\end{figure}
\subsubsection*{Documentazione}
Di seguito la specifica openapi dell'endpoint register contenuta nel file 'openapi.yaml' della directory del microservizio
\begin{verbatim}
	/register:
    post:
      summary: Registering method for new users
      description: This endpoint allows new users to register by providing email, password, and two-factor authentication code.
      requestBody:
        required: true
        content:
          application/x-www-form-urlencoded:
            schema:
              type: object
              properties:
                email:
                  type: string
                  format: email
                  
                password:
                  type: string
                  format: password
                twofa:
                  type: string
                  format: twofa-code
              required:
                - email
                - password
                - twofa
      responses:
        '200':
          description: Succesfully registered
          content:
            application/json:
              schema:
                type: object
                properties:
                  text:
                    type: string
                    example: "Successfully registered"
                  token:
                    type: string
                    example: "1231345321532151"
          headers:
            Set-Cookie:
              schema:
                type: string
              description: Session token
        '400':
          description: Bad Request - Missing fields or 2fa not correct or User already exists
          content:
            application/json:
              schema:
                oneOf:
                  - $ref: '#/components/schemas/missingFieldsSchema'
                  - $ref: '#/components/schemas/wrongTwoFaSchema'
                  - $ref: '#/components/schemas/alreadyExistsSchema'

\end{verbatim}
\subsubsection*{Testing}
\begin{figure}[H]
	\centering\includegraphics[width=1\textwidth]{images/microservizio-autenticazione/tests/register_test_1.png}
	\caption{Test Register 1}
\end{figure}
\begin{figure}[H]
	\centering\includegraphics[width=1\textwidth]{images/microservizio-autenticazione/tests/register_test_2.png}
	\caption{Test Register 2}
\end{figure}
\begin{figure}[H]
	\centering\includegraphics[width=1\textwidth]{images/microservizio-autenticazione/tests/register_test_3.png}
	\caption{Test Register 3}
\end{figure}
\subsection{Cambio password}
\subsubsection*{Specifica}
Questa API, all'indirizzo /api/auth/changepass, ha un metodo PATCH e viene utilizzata per cambiare la password di un utente.
La request body è in formato x-www-form-urlencoded e deve contenere i seguenti campi:
\begin{itemize}
	\item email
	\item new\_password
	\item twofa
\end{itemize}
La response body è in formato application/json. Di seguito le possibili risposte:
% table
\begin{center} % center the table
	\centering
	\begin{tabular}{ |p{4cm}|p{5cm}|p{4cm}| }
		\hline
		\centering Status Code & \qquad\quad Body e Cookie                                           & \qquad\qquad Spiegazione                              \\ % I found no other way...
		\hline
		200 OK                 & \{text: "successfully changed your password", token\} cookie: token & Cambio password avvenuto con successo                 \\
		\hline
		400 BAD REQUEST        & \{error: "missing fields", missingFields\}                          & Alcuni campi non sono stati specificati nella request \\ % I found no other way...
		\hline
		404 NOT FOUND          & \{error: "user not found"\}                                         & non esiste utente con la mail fornita                 \\% I found no other way...
		\hline
		400 BAD REQUEST        & \{error: "2fa not correct" \}                                       & Il codice twofa inserito non è corretto.              \\
		\hline
	\end{tabular}
\end{center}
\begin{figure}[H]
	\centering\includegraphics[width=1\textwidth]{images/microservizio-autenticazione/diagrams/diagramma_changepass.drawio.png}
	\caption{Diagramma dell'endpoint "changepass"}
\end{figure}
\subsubsection*{Sviluppo}
Questa API viene utilizzata da un utente per cambiare la password del proprio account.
Una volta ricevuta la richiesta, viene controllata la presenza di tutti i campi richiesti.
Successivamente si controlla se l'utente esista o meno nel database.
Dopo aver controllato che il codice 2FA sia corretto,viene modificata la password nel database, viene generato un token, aggiunto alla sessione e inviato sia nel body sia come cookie nella risposta.

\begin{figure}[H]
	\centering\includegraphics[width=1\textwidth]{images/microservizio-autenticazione/changepass-carbon.png}
	Codice dell'API api/auth/changepass
\end{figure}
\subsubsection*{Documentazione}
Di seguito la specifica openapi dell'endpoint changepass contenuta nel file 'openapi.yaml'
\begin{verbatim}
	/changepass:
    patch:
      summary: Password Changing Method
      description: This endpoint allows existing users to change their passwords  by providing email, the new password, and two-factor authentication code.
      requestBody:
        required: true
        content:
          application/x-www-form-urlencoded:
            schema:
              type: object
              properties:
                email:
                  type: string
                  format: email
                new_password:
                  type: string
                  format: password
                twofa:
                  type: string
                  format: twofa-code
              required:
                - email
                - new_password
                - twofa
      responses:
        '200':
          description: Successfully changed your password
          content:
            application/json:
              schema:
                type: object
                properties:
                  text:
                    type: string
                    example: "successfully changed your password"
                  token:
                    type: string
                    example: "12312434325321234"
          headers:
            Set-Cookie:
              schema:
                type: string
              description: Session token
        '400':
          description: Bad Request - Missing fields or 2fa not correct
          content:
            application/json:
              schema:
                oneOf:
                  - $ref: '#/components/schemas/missingFieldsSchema'
                  - $ref: '#/components/schemas/wrongTwoFaSchema'
      
        '404':
          description: User not found
          content:
            application/json:
              schema:
                $ref: '#/components/schemas/notFoundSchema'

\end{verbatim}
\subsubsection*{Testing}
\begin{figure}[H]
	\centering\includegraphics[width=1\textwidth]{images/microservizio-autenticazione/tests/password_test_1.png}
	\caption{Test Pass 1}
\end{figure}
\begin{figure}[H]
	\centering\includegraphics[width=1\textwidth]{images/microservizio-autenticazione/tests/password_test_2.png}
	\caption{Test Pass 2}
\end{figure}
\subsection{Eliminazione Profilo}
\subsubsection*{Specifica}
Questa API, all'indirizzo /api/auth/remove, ha un metodo POST e viene utilizzata per rimuovere il profilo di un utente.
La request body è in formato x-www-form-urlencoded e deve contenere i seguenti campi:
\begin{itemize}
	\item email
	\item password
	\item twofa
\end{itemize}
La response body è in formato application/json. Di seguito le possibili risposte:
% table
\begin{center} % center the table
	\centering
	\begin{tabular}{ |p{4cm}|p{5cm}|p{4cm}| }
		\hline
		\centering Status Code & \qquad\quad Body e Cookie                                          & \qquad\qquad Spiegazione                              \\ % I found no other way...
		\hline
		200 OK                 & \{text: "successfully removed your account", token\} cookie: token & L'account è stato rimosso con successo                \\
		\hline
		400 BAD REQUEST        & \{error: "missing fields", missingFields\}                         & Alcuni campi non sono stati specificati nella request \\ % I found no other way...
		\hline
		404 NOT FOUND          & \{error: "user not found"\}                                        & non esiste utente con la mail fornita                 \\% I found no other way...
		\hline
		404 NOT FOUND          & \{error: "wrong password" \}                                       & la password inserita non è corretta                   \\
		\hline
		400 BAD REQUEST        & \{error: "2fa not correct" \}                                      & Il codice twofa inserito non è corretto.              \\
		\hline
	\end{tabular}
\end{center}
\begin{figure}[H]
	\centering\includegraphics[width=1\textwidth]{images/microservizio-autenticazione/diagrams/diagramma_remove.drawio.png}
	\caption{Diagramma dell'endpoint "remove"}
\end{figure}
\subsubsection*{Sviluppo}
Questa API viene utilizzata da un utente per eliminare il proprio profilo.
Una volta ricevuta la richiesta, viene controllata la presenza di tutti i campi richiesti.
Successivamente si controlla se l'utente esista o meno nel database, e successivamente la validità della password.
Dopo aver controllato che il codice 2FA sia corretto, il profilo viene eliminato dal database e il suo token rimosso dalla sessione.


\begin{figure}[H]
	\centering\includegraphics[width=1\textwidth]{images/microservizio-autenticazione/remove-carbon.png}
	Codice dell'API api/auth/remove
\end{figure}
\subsubsection*{Documentazione}
Di seguito la specifica openapi dell'endpoint remove contenuta nel file 'openapi.yaml' della directory del microservizio

\begin{verbatim}
	/remove:
    delete:
      summary: Account removal method
      description: This endpoint allows existing users to remove their accounts by providing email, password, and two-factor authentication code.
      requestBody:
        required: true
        content:
          application/x-www-form-urlencoded:
            schema:
              type: object
              properties:
                email:
                  type: string
                  format: email
                password:
                  type: string
                  format: password
                twofa:
                  type: string
                  format: twofa-code
              required:
                - email
                - password
                - twofa
      responses:
        '200':
          description: Succesfully removed the account
          content:
            application/json:
              schema:
                type: object
                properties:
                  text:
                    type: string
                    example: "Successfully removed your account."
        '400':
          description: Bad Request - Missing fields or 2fa not correct
          content:
            application/json:
              schema:
                oneOf:
                  - $ref: '#/components/schemas/missingFieldsSchema'
                  - $ref: '#/components/schemas/wrongTwoFaSchema'
      
        '404':
          description: User not found or wrong password
          content:
            application/json:
              schema:
                oneOf:
                  - $ref: '#/components/schemas/notFoundSchema'
                  - $ref: '#/components/schemas/wrongPassSchema'

\end{verbatim}
\subsubsection*{Testing}
\begin{figure}[H]
	\centering\includegraphics[width=1\textwidth]{images/microservizio-autenticazione/tests/remove_test_1.png}
	\caption{Test Remove 1}
\end{figure}
\begin{figure}[H]
	\centering\includegraphics[width=1\textwidth]{images/microservizio-autenticazione/tests/remove_test_2.png}
	\caption{Test Remove 2}
\end{figure}
\begin{figure}[H]
	\centering\includegraphics[width=1\textwidth]{images/microservizio-autenticazione/tests/remove_test_3.png}
	\caption{Test Remove 3}
\end{figure}
\subsection{Autentifica Token}
\subsubsection*{Specifica}
Questa API, all'indirizzo/api/auth/authenticate, ha un metodo POST e viene utilizzata da un altro microservizio per autenticare un utente utilizzando il suo token.
La request body è in formato x-www-form-urlencoded e deve contenere il token dell'utente.
La response body è in formato application/json. di seguito le possibili risposte:
% table
\begin{center} % center the table
	\centering
	\begin{tabular}{ |p{4cm}|p{5cm}|p{4cm}| }
		\hline
		\centering Status Code & \qquad\quad Body e Cookie                            & \qquad\qquad Spiegazione                                                                                                     \\ % I found no other way...
		\hline
		200 OK                 & \{text: "success", user\_info: \{permission, id\} \} & L'utente esiste, e ha il seguente id e permissionLevel                                                                       \\
		\hline
		400 BAD REQUEST        & \{error: "missing fields", missingFields\}           & Alcuni campi non sono stati specificati nella request                                                                        \\ % I found no other way...
		\hline
		404 NOT FOUND          & \{error: "user not found with the given token"\}     & non esiste utente nella sessione con il token fornito.                                                                       \\% I found no other way...
		\hline
		404 NOT FOUND          & \{error: "user doesn't exist or is deleted" \}       & Caso impossibile a meno di bug: il token esiste nella sessione, ma l'id utente non corrisponde ad alcun utente del database. \\
		\hline
	\end{tabular}
\end{center}
\begin{figure}[H]
	\centering\includegraphics[width=1\textwidth]{images/microservizio-autenticazione/diagrams/diagramma_auth.drawio.png}
	\caption{Diagramma dell'endpoint "authenticate"}
\end{figure}
\subsubsection*{Sviluppo}
Questa API viene utilizzata da un microservizio per autenticare un utente che desidera farne uso.
Una volta ricevuta la richiesta, viene controllata la presenza di tutti i campi richiesti.
Successivamente si controlla se il token fornito abbia un utente corrispondente nella sessione.
Infine il livello di permesso associato all'utente e il suo id vengono inviati nella risposta.
\begin{figure}[H]
	\centering\includegraphics[width=1\textwidth]{images/microservizio-autenticazione/authenticate-carbon.png}
	Codice dell'API api/auth/authenticate
\end{figure}
\subsubsection*{Documentazione}
Di seguito la specifica openapi dell'endpoint authenticate contenuta nel file 'openapi.yaml' della directory del microservizio
\begin{verbatim}
	/authenticate:
    post:
      summary: Method for authentication 
      description: This endpoint allows other microservices to authenticate a user by providing their token
      requestBody:
        required: true
        content:
          application/x-www-form-urlencoded:
            schema:
              type: object
              properties:
                token:
                  type: string
                  
              required:
                  - token
      responses:
        '200':
          description: Success
          content:
            application/json:
              schema:
                type: object
                properties:
                  text:
                    type: string
                    example: "success"
                  user_info:
                    type: object
                    properties:
                      permission:
                        type: string
                        example: "Manager"
                      id:
                        type: string
                        example: "aa231e3421bd1223"

      

        '400':
          description: Bad Request - Missing fields 
          content:
            application/json:
              schema:
                $ref: '#/components/schemas/missingFieldsSchema'
        '404':
          description: User not found with the given token or user doesn't exist 
          content:
            application/json:
              schema:
                oneOf:
                  - $ref: '#/components/schemas/notFoundSchema'
                  - $ref: '#/components/schemas/maybeDeletedSchema'

\end{verbatim}

\subsection{Two Factor Authentication}
\subsubsection*{Specifica}
Questa API, all'indirizzo/api/auth/twofa, ha un metodo POST e viene utilizzata da un utente per richiedere un codice 2FA attraverso mail
La request body è in formato x-www-form-urlencoded e deve contenere l'indirizzo email dell'utente.
La response body è in formato application/json. di seguito le possibili risposte:
% table
\begin{center} % center the table
	\centering
	\begin{tabular}{ |p{4cm}|p{5cm}|p{4cm}| }
		\hline
		\centering Status Code & \qquad\quad Body e Cookie                                & \qquad\qquad Spiegazione                              \\ % I found no other way...
		\hline
		200 OK                 & \{text: "Successfully sent the 2FA code to your email"\} & Il codice è stato inviato alla mail correttamente     \\
		\hline
		400 BAD REQUEST        & \{error: "missing fields", missingFields\}               & Alcuni campi non sono stati specificati nella request \\ % I found no other way...
		\hline
	\end{tabular}
\end{center}
\begin{figure}[H]
	\centering\includegraphics[width=1\textwidth]{images/microservizio-autenticazione/diagrams/diagramma_twofa.drawio.png}
	\caption{Diagramma dell'endpoint "changepass"}
\end{figure}
\subsubsection*{Sviluppo}
Questa API viene utilizzata da un utente per ricevere un codice 2FA attraverso email.
Una volta ricevuta la richiesta, viene controllata la presenza di tutti i campi richiesti.
Successivamente viene inviata l'email con il codice 2FA.

\begin{figure}[H]
	\centering\includegraphics[width=1\textwidth]{images/microservizio-autenticazione/twofa-carbon.png}
	Codice dell'API api/auth/twofa
\end{figure}

\subsubsection*{Documentazione}
Di seguito la specifica openapi dell'endpoint twofa contenuta nel file 'openapi.yaml' della directory del microservizio
\begin{verbatim}
	/twofa:
    post:
      summary: Method for receiving a 2FA code 
      description: This endpoint allows users to request a Two-Factor Authentication code by providing their email
      requestBody:
        required: true
        content:
          application/x-www-form-urlencoded:
            schema:
              type: object
              properties:
                token:
                  type: string
              required:
                  - token
      responses:
        '200':
          description: Succesfully sent the 2FA
          content:
            application/json:
              schema:
                type: object
                properties:
                  text:
                    type: string
                    example: "Successfully sent the 2FA code to your email(SPOILER: IT'S 12345)"
        '400':
          description: Bad Request - Missing fields 
          content:
            application/json:
              schema:
                $ref: '#/components/schemas/missingFieldsSchema'
                     
     
\end{verbatim}
\subsubsection*{Testing}
\begin{figure}[H]
	\centering\includegraphics[width=1\textwidth]{images/microservizio-autenticazione/tests/twofa_test_1.png}
	\caption{TwoFa Testing 1}
\end{figure}
\begin{figure}[H]
	\centering\includegraphics[width=1\textwidth]{images/microservizio-autenticazione/tests/twofa_test_2.png}
	\caption{TwoFa Testing 2}
\end{figure}
\section{Microservizio Task}

\section{Microservizio Gestione Dipendenti}
	\subsection*{Struttura del backend}
	\subsection*{Modellazione dati nel database}
	\subsection*{Specifica delle Risorse}

	\subsection*{Crea profilo Dipendente}

		\subsubsection*{Specifica}
		Questa API, all'indirizzo /api/dipendenti/create, ha un metodo POST e viene utilizzata per la creazione di un profilo dipendente da parte del Manager.
		La request body è in formato x-www-form-urlencoded e deve contenere i seguenti campi:
		\begin{itemize}
			
			\item email
			\item password
			\item data di nascita
			\item data di assunzione
			\item lista worktag
		\end{itemize}
		Inoltre, per verificare che il manager sia autenticato, il suo token va inserito o all'interno del body o attraverso un cookie.
		La response body è in formato application/json. Di seguito le possibili risposte:
		% table
		\begin{center} % center the table
			\centering
			\begin{tabular}{ |p{4cm}|p{5cm}|p{4cm}| }
				\hline
				\centering Status Code & \qquad\quad Body e Cookie                                & \qquad\qquad Spiegazione                              \\ % I found no other way...
				\hline
				200 OK                 & \{text: "successfully created the profile"\} 			  & Profilo dipendente creato con successo                           \\
				\hline
				401 UNAUTHORIZED       & \{error: "missing token"\} 							  & Il token non è presente nella richiesta				  \\
				\hline
				403 FORBIDDEN		   & \{error: "permission denied"\}							  & Il token è presente nella richiesta, ma l'utente associato non è un manager \\	
				\hline
				404 NOT FOUND 		   & \{error: "No user found with the given token"\}          & Il token è presente nella richiesta, ma non corrisponde a nessun profilo autenticato \\
				\hline
				400 BAD REQUEST        & \{error: "missing fields", missingFields\}               & Alcuni campi non sono stati specificati nella request \\ % I found no other way...
				\hline
				400 BAD REQUEST        & \{error: "user already exists with the given email"\}    & Esiste già un utente con la mail fornita              \\% I found no other way...
				\hline
				400 BAD REQUEST        & \{error: "invalid birth date" \}                            & La data di nascita fornita non è in un formato riconosciuto             \\
				\hline
				400 BAD REQUEST        & \{error: "invalid assunzione date" \}                            & La data di assunzione fornita non è in un formato riconosciuto             \\
				\hline
				400 BAD REQUEST        & \{error: "no worktag array" \}                            & il campo della richiesta 'worktag' non è un array           \\
				\hline
				400 BAD REQUEST        & \{error: "elements provided are not worktags" \}                            & Gli elementi forniti all'interno di 'worktag' non sono worktags           \\
				\hline
			\end{tabular}
		\end{center}
		\subsubsection*{Sviluppo}

		\subsubsection*{Documentazione}

		\subsubsection*{Testing}

	\subsection*{Elimina profilo}
		\subsubsection*{Specifica}
		Questa API, all'indirizzo /api/dipendenti/delete, ha un metodo delete e viene utilizzata per l'eliminazione di un profilo da parte del Manager.
		La request body è in formato x-www-form-urlencoded e deve contenere i seguenti campi:
		\begin{itemize}
			\item email
		\end{itemize}
		Inoltre, per verificare che il manager sia autenticato, il suo token va inserito o all'interno del body o attraverso un cookie.
		La response body è in formato application/json. Di seguito le possibili risposte:
		% table
		\begin{center} % center the table
			\centering
			\begin{tabular}{ |p{4cm}|p{5cm}|p{4cm}| }
				\hline
				\centering Status Code & \qquad\quad Body e Cookie                                & \qquad\qquad Spiegazione                              \\ % I found no other way...
				\hline
				200 OK                 & \{text: "successfully deleted the profile"\} 			  & Profilo eliminato con successo                           \\
				\hline
				401 UNAUTHORIZED       & \{error: "missing token"\} 							  & Il token non è presente nella richiesta				  \\
				\hline
				403 FORBIDDEN		   & \{error: "permission denied"\}							  & Il token è presente nella richiesta, ma l'utente associato non è un manager \\	
				\hline
				404 NOT FOUND 		   & \{error: "No user found with the given token"\}          & Il token è presente nella richiesta, ma non corrisponde a nessun profilo autenticato \\
				\hline
				400 BAD REQUEST        & \{error: "missing fields", missingFields\}               & Alcuni campi non sono stati specificati nella request \\ % I found no other way...
				\hline
				404 NOT FOUND 		   & \{error: "no user with the given email" \}				  & Non esiste alcun utente con la mail fornita      \\
				\hline
			\end{tabular}
		\end{center}
		\subsubsection*{Sviluppo}

		\subsubsection*{Documentazione}

		\subsubsection*{Testing}

	\subsection*{Cerca profili}
		\subsubsection*{Specifica}
		Questa API, all'indirizzo /api/dipendenti/find, ha un metodo post e viene utilizzata per la ricerca di uno o più profili da parte del manager.
		La request body è in formato x-www-form-urlencoded e deve contenere i seguenti campi:
		\begin{itemize}
			\item mode
		\end{itemize}
		il campo mode può assumere i valori 'one' o 'many', e indica se la ricerca debba ritornare uno o più profili.
		Il body può contenere, ma non è obbligatorio, i seguenti campi per una ricerca più specifica:
		\begin{itemize}
			\item email
			\item nome
			\item cognome
			\item data di nascita
			\item data di assunzione
			\item worktags
		\end{itemize}
		Inoltre, per verificare che il manager sia autenticato, il suo token va inserito o all'interno del body o attraverso un cookie.
		La response body è in formato application/json. Di seguito le possibili risposte:
		% table
		\begin{center} % center the table
			\centering
			\begin{tabular}{ |p{4cm}|p{5cm}|p{4cm}| }
				\hline
				\centering Status Code & \qquad\quad Body e Cookie                                & \qquad\qquad Spiegazione                              \\ % I found no other way...
				\hline
				200 OK                 & \{user: {...}\} 			  & La ricerca con mode = "one" ha riportato un risultato                           \\
				\hline
				200 OK 				   & \{users: [user1,user2,user3...]\} & La ricerca con mode = 'many' ha riportato uno o più risultati              \\
				\hline
				401 UNAUTHORIZED       & \{error: "missing token"\} 							  & Il token non è presente nella richiesta				  \\
				\hline
				403 FORBIDDEN		   & \{error: "permission denied"\}							  & Il token è presente nella richiesta, ma l'utente associato non è un manager \\	
				\hline
				404 NOT FOUND 		   & \{error: "No user found with the given token"\}          & Il token è presente nella richiesta, ma non corrisponde a nessun profilo autenticato \\
				\hline
				400 BAD REQUEST        & \{error: "missing fields", missingFields\}               & Alcuni campi non sono stati specificati nella request \\ % I found no other way...
				\hline
				404 NOT FOUND 		   & \{error: "no user found" \}				  & Non esiste alcun utente avente i parametri di ricerca forniti. viene ritornato solo con mode = one   \\
				\hline
			\end{tabular}
		\end{center}
		\subsubsection*{Sviluppo}

		\subsubsection*{Documentazione}

		\subsubsection*{Testing}

	\subsection*{Storico profili}
		\subsubsection*{Specifica}
		Questa API, all'indirizzo /api/dipendenti/history, ha un metodo post e viene utilizzata per la visione dello storico task di un determinato dipendente
		La request body è in formato x-www-form-urlencoded e deve contenere i seguenti campi:
		\begin{itemize}
			\item email
		\end{itemize}
		Inoltre, per verificare che il manager sia autenticato, il suo token va inserito o all'interno del body o attraverso un cookie.
		La response body è in formato application/json. Di seguito le possibili risposte:
		% table
		\begin{center} % center the table
			\centering
			\begin{tabular}{ |p{4cm}|p{5cm}|p{4cm}| }
				\hline
				\centering Status Code & \qquad\quad Body e Cookie                                & \qquad\qquad Spiegazione                              \\ % I found no other way...
				\hline
				200 OK                 & \{[task1,task2,task3...]\} 			  & La ricerca ha riportato lo storico task con successo                          \\
				\hline
				401 UNAUTHORIZED       & \{error: "missing token"\} 							  & Il token non è presente nella richiesta				  \\
				\hline
				403 FORBIDDEN		   & \{error: "permission denied"\}							  & Il token è presente nella richiesta, ma l'utente associato non è un manager \\	
				\hline
				404 NOT FOUND 		   & \{error: "No user found with the given token"\}          & Il token è presente nella richiesta, ma non corrisponde a nessun profilo autenticato \\
				\hline
				400 BAD REQUEST        & \{error: "missing fields", missingFields\}               & Alcuni campi non sono stati specificati nella request \\ % I found no other way...
				\hline
				404 NOT FOUND 		   & \{error: "dipendente not foudn" \}				  & Non esiste alcun dipendente avente la mail fornita \\
				\hline
			\end{tabular}
		\end{center}
		\subsubsection*{Sviluppo}

		\subsubsection*{Documentazione}

		\subsubsection*{Testing}

\section{Microservizio Home}

\chapter{Frontend}


\end{document}
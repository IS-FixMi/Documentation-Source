\documentclass{report}
\usepackage{float}

% depricated
\title{
	\iffalse
	\begin{tikzpicture}[remember picture,overlay]
		\node[anchor=north west,yshift=-5pt,xshift=-200pt]%
		at (current page.north east)
		{\includegraphics[height=30mm]{unitn-logo}};
	\end{tikzpicture}
	\huge 
	\fi
	Fix Mi \\
	Analisi dei Requisiti
}
\author{Giovanni Santini, Riginel Ungureanu, Valerio Asaro}
\date{Anno accademico 2023/2024}

% for the image in the title
\usepackage{tikz}

% custom spacing
\usepackage{setspace}
\onehalfspacing

% footer and header
\usepackage{fancyhdr}
% \setlength{\headheight}{15.2pt}

% Table of contents link to corresponding sections
\usepackage{hyperref}
\hypersetup{
	colorlinks,
	citecolor=black,
	filecolor=black,
	linkcolor=black,
	urlcolor=black
}

\usepackage{amsmath}
% Remove che "Chapter" string before chapters
\iffalse
\makeatletter
\def\@makechapterhead#1{%
	\vspace*{50\p@}%
	{\parindent \z@ \raggedright \normalfont
		\interlinepenalty\@M
		\Huge\bfseries  \thechapter.\quad #1\par\nobreak
		\vskip 40\p@
}}
\makeatother
\fi

% Fancy chapters
\usepackage[Bjarne]{fncychap}
% options: Sonny, Lenny, Glenn, Conny, Rejne, Bjarne, Bjornstrup

\begin{document}
	
	
	%title page
	\begin{titlepage}
		\begin{figure}[t]
			\centering\includegraphics[width=0.3\textwidth]{images/unitn-logo}
		\end{figure}
		\begin{center}
			\textsc{ \LARGE{Università degli Studi di Trento \\}}
			\textsc{ \LARGE{Facoltà di Informatica\\ }}
			\textnormal{ \LARGE{Corso di Ingegneria del Software\\}}
			\vspace{30mm}
			\fontsize{10mm}{7mm}\selectfont 
			\textup{Fix Mi \\ Specifica dei Requisiti}\\
		\end{center}
		
		\vspace{25mm}
		
		\centering
		\large Giovanni Santini\\ Riginel Ungureanu \\ Valerio Asaro
		
		\vspace{20mm}
		
		\centering{\large{Anno Accademico 2023/2024 \\ Trento }}
		
	\end{titlepage}
	
	
	
	
	% use header and footers
	\pagestyle{fancy}
	\fancyhead[R]{\chaptername\ \thechapter}  % header
	
	%\maketitle
	\tableofcontents
	\newpage
	
	
	
	\section{Scopo del documento}
	
	Nel presente documento vengono riportate le specifiche dei requisiti di sistema del progetto FixMi,  attraverso diagrammi di tipo Unified Modeling Language (UML) e tabelle strutturate.\\
	
	
	
	\section{Informazioni del Documento}
	
	% table
	\begin{center} % center the table
		\centering
		\begin{tabular}{ |p{4cm}|p{4cm}|  }
			\hline
			\centering Campo & \qquad\qquad Valore \\ % I found no other way...
			\hline
			Titolo del Documento & Specifica dei Requisiti \\
			\hline
			Titolo del Progetto & Fix Mi \\
			\hline
			Autori del Documento &
			Giovanni Santini \\ & Riginel Ungureanu \\ & Valerio Asaro \\
			\hline
			Amministratore Progetto & Riginel Ungureanu\\
			\hline
			Versione del documento & 1.0 \\
			\hline
		\end{tabular}
	\end{center}


\chapter{Requisiti}

	
\section{Requisiti Funzionali}

\begin{itemize}
	\item Use Case Diagram: Visione esterna del sistema
	\item Sequence Diagram: Rappresenta come gli oggetti collaborano
	\item State Machine Diagram: Stati e Transizioni
	\item Activity Diagram: Attività che innescano altre attività (tasks)
	\item Spiegazione in italiano (da mettere sempre)
\end{itemize}

\subsection*{RF1 Login }
\begin{figure}[H]
	\centering\includegraphics[width=1\textwidth]{images/Login_UCD.drawio.png}
	Use Case Diagram  del login
\end{figure}
Per descrivere questo use case, facciamo uso di un diagramma delle attività swimlane:
\begin{figure}[H]
	\centering\includegraphics[width=1\textwidth]{images/Login_Swimlane.drawio.png}
	diagramma delle Attività swimlane del Login
\end{figure}
\subsection*{RF2 Registrazione}
\begin{figure}[H]
	\centering\includegraphics[width=1\textwidth]{images/Registrazione_UCD.drawio.png}
	Use Case Diagram  della registrazione
\end{figure}
Per descrivere questo use case, facciamo uso di un diagramma delle attività swimlane:
\begin{figure}[H]
	\centering\includegraphics[width=1\textwidth]{images/Register_swimlane.drawio.png}
	diagramma delle Attività swimlane della registrazione
\end{figure}

\subsection{RF4 About Contatti}

\begin{figure}[H]
	\centering\includegraphics[width=1\textwidth]{images/contatti_sequence_diagram.png}
\end{figure}


\subsection*{RF3 Negozio utente non autenticato + RF5 Negozio utente autenticato}

\begin{figure}[H]
	\centering\includegraphics[width=1\textwidth]{images/shop_diagram_1.png}
\end{figure}

\begin{figure}[H]
	\centering\includegraphics[width=1\textwidth]{images/negozio_sequence_diagram.png}
	diagramma delle Attività swimlane della registrazione
\end{figure}

\subsection*{RF6 Riparazione}
\begin{figure}[H]
	\centering\includegraphics[width=1\textwidth]{images/Riparazione_UCD.drawio.png}
	Use Case Diagram  della riparazione
\end{figure}
\subsubsection*{Descrizione Use Case "Riparazione"}
\textbf{Titolo}: Riparazione \newline
\textbf{Riassunto}: Questo use case descrive come un utente può richiedere una riparazione e visualizzarne lo stato. \newline
\textbf{Descrizione}:
	\begin{enumerate}
		\item L'utente autenticato seleziona la pagina "Riparazioni";
		\item Il sito mostra lo stato delle riparazioni già richieste dall'utente [Exception 1];
		\item Il sito mostra un form per richiedere una nuova riparazione con i seguenti campi:
		\begin{itemize}
			\item nome[Exception 2];
			\item cognome[Exception 2]; 
			\item email[Exception 2];
			\item numero di telefono[Exception 2];
			\item descrizione del problema[Exception 2];
			\item foto (( facoltativo ));
		\end{itemize}
		\item L'utente, appena compilato il form, può inviarlo premendo l'apposito pulsante;
		\item Il sistema, appena ricevuta la richiesta di riparazione, la inserisce nel sistema delle task come "task riparazione";
		
	\end{enumerate}
\textbf{Exceptions}
\begin{itemize}
	\item {[Exception 1]}: Se L'utente non ha nessuna riparazione richiesta, l'elenco sarà vuoto;
	\item {[ Exception 2]}: Se l'utente non ha compilato i campi "nome","cognome","email","numero di telefono","descrizione del problema" non può inviare la richiesta;
\end{itemize}
\subsection*{RF7 Assistenza}

\begin{figure}[H]
	\centering\includegraphics[width=1\textwidth]{images/assistenza_UCD.png}
\end{figure}


\subsection*{RF8 Feedback}

\begin{figure}[H]
	\centering\includegraphics[width=0.9\textwidth]{images/feedback_UCD.png}
\end{figure}

\subsection*{RF9 Tasks}

\subsection*{RF10 Magazzino}
\centering\includegraphics[width=1\textwidth]{images/magazzino_UCD.png}

\subsection*{RF11 Gestione Dipendenti}
\centering\includegraphics[width=1\textwidth]{images/gestione_dipendenti_UCD.png}  % Non riesco a fare in modo che l'immagine si veda abbastanza grande, probabilmente il grafico stesso è troppo grande.....lo aggiusterò in seguito....credo....probabilmente c'ò solo da posizionare gli elementi in maniera corretta....

\pagebreak

\section{Requisiti Non Funzionali}
Nel seguente capitolo vengono riportati i requisiti non funzionali (RNF) del sistema utilizzando tabelle strutturate e specificando misure facilmente misurabili


\subsection*{RNF1 Intuitività e Accessibilità}
% table
\begin{center} % center the table
	\centering
	\begin{tabular}{ |p{3cm}|p{4cm}|p{4cm}|  }
		\hline
		\centering Proprietà & \qquad\quad Descrizione & \qquad\qquad Misura\\ % I found no other way...
		\hline
		Linguaggio Comprensibile & In media l’utente deve essere in grado di capire le funzionalità dell'applicazione con una
		sola lettura della descrizione & Il numero di click sbagliati che l'utente fa deve essere minore di 4\\
		\hline
		Presenza della lingua inglese e italiana & Il sito presenta sia la lingua Italiana che quella inglese, l'utente con un livello di lingua A1 è in grado di leggere e comprendere il contenuto & Certificato linguistico \\ % I found no other way...
		\hline
		Consistenza & Il sito deve avere un design consistente, utilizzando un singolo font e una palette fissa di colori & numero di font utilizzati, numero di colori utilizzati  \\% I found no other way... 
		\hline
	\end{tabular}
\end{center}

\subsection*{RNF2 Sicurezza}
% table
\begin{center} % center the table
	\centering
	\begin{tabular}{ |p{3cm}|p{4cm}|p{4cm}|  }
		\hline
		\centering Proprietà & \qquad\quad Descrizione & \qquad\qquad Misura\\ % I found no other way...
		\hline
		Protezione dati & Il sito deve proteggere i dati sensibili: 
		\begin{itemize}
			\item Utilizzo di hashing SHA-3 per le password;
			\item Utilizzo dei protocolli tls e https per ogni comunicazione tra utenti e servizi;
		\end{itemize} & \\
		\hline
		2 Factor Authentication & Il sito deve verificare l'identità dell'utente attraverso il 2FA & \\
		\hline
		Conformità password & La password di un utente deve avere una lunghezza minima di 10 caratteri e deve presentare almeno un numero,una lettera maiuscola, e un carattere speciale &
		numero dei caratteri, presenza di numeri,lettere maiuscole e caratteri speciali della lista:		\begin{verbatim} ! ? $ % ^ & * ( ) _ \end{verbatim} \begin{verbatim}- + = { [ } ] : ;\end{verbatim} \begin{verbatim}@ # | \ < , > . \end{verbatim} \\
		\hline
		
	\end{tabular}
\end{center}
\subsection*{RNF3 Privacy}
% table
\begin{center} % center the table
	\centering
	\begin{tabular}{ |p{3cm}|p{4cm}|p{4cm}|  }
		\hline
		\centering Proprietà & \qquad\quad Descrizione & \qquad\qquad Misura\\ % I found no other way...
		\hline
		GDPR & il sito deve essere conforme alle principali direttive del GDPR, tra cui il consenso esplicito per la raccolta dei dati, la trasparenza nell'uso dei dati, la possibilità di accesso e cancellazione dei dati personali da parte dell'individuo, e misure di sicurezza per proteggere tali dati & Conforme \\
		\hline
	\end{tabular}
\end{center}
\subsection*{RNF4 Affidabilità e Disponibilità}
% table
\begin{center} % center the table
	\centering
	\begin{tabular}{ |p{3cm}|p{4cm}|p{4cm}|  }
		\hline
		\centering Proprietà & \qquad\quad Descrizione & \qquad\qquad Misura\\ % I found no other way...
		\hline
		Risultati desiderati & La probabilità che il sito fornisca i risultati desiderati senza interruzioni o tempi di inattività deve essere maggiore del 99\% (novantanove percento) & $\frac{\text{risultati ricevuti con successo}}{\text{risultati totali}} $ \\ % There was no other way...
		\hline
		Operatività &  la probabilità che il sito rimanga operativo in un determinato momento
		indipendentemente dal numero di guasti già subiti dal sistema deve essere maggiore del 99\% (novantanove percento) & $ \frac{\text{secondi di attività dal lancio}}{\text{secondi totali dal lancio}} $\\
		\hline 
	\end{tabular}
\end{center}
\subsection*{RNF5 Performante}
% table
\begin{center} % center the table
	\centering
	\begin{tabular}{ |p{3cm}|p{4cm}|p{4cm}|  }
		\hline
		\centering Proprietà & \qquad\quad Descrizione & \qquad\qquad Misura\\ % I found no other way...
		\hline
		Aggiornamento negozio & Il sito deve aggiornare la lista degli articoli presenti in negozio, in caso di modifica al magazzino, in meno di un secondo. & Secondi \\
		\hline 
		2 Factor Authentication (2FA)& Il sito deve inviare la mail di 2FA in meno di cinque secondi. & Secondi \\
		\hline
		Lista delle Task & Il sito deve aggiornare la lista delle task in meno di un secondo. & Secondi \\ 
		\hline

	\end{tabular}
\end{center}
\subsection*{RNF6 Compatibilità e Portabilità}
% table
\begin{center} % center the table
	\centering
	\begin{tabular}{ |p{3cm}|p{4cm}|p{4cm}|  }
		\hline
		\centering Proprietà & \qquad\quad Descrizione & \qquad\qquad Misura\\ % I found no other way...
		\hline
		Compatibilità dispositivi lato client & L'applicazione lato cliente  deve essere disponibile per dispositivi aventi un browser che supporta \begin{itemize}
		\item html5
		\item https
		\item tls 1.2
		\end{itemize} & \\
		\hline
		Compatibilità dispositivi lato server & L'applicazione lato server deve essere disponibile per computer che supportino: 
		\begin{itemize}
			\item Node js 18.18.0 LTS
			\item MongoDB 7.0
		\end{itemize}
		& \\
		\hline 
		Responsive su TV e monitor di PC e Laptop & Il sito deve potersi adattare alla dimensione degli schermi con Aspect Ratio da 4:3, 16:9, 21:9 & Aspect Ratio \\
		\hline
		Responsive su Smartphone & Il sito deve potersi adattare agli schermi dei seguenti Smartphone: 
		\begin{itemize}
			\item Iphone X,XR,11,\dots , 14
			\item Tutti i modelli Xiaomi dal 2018 in poi 
			\item Tutti i modelli Samsung dal 2018 in poi
			\item Tutti i modelli Motorola dal 2018 in poi 
			\item Tutti i modelli Huawei dal 2018 in poi 
		\end{itemize}
		& \\
		
		\hline
		
	\end{tabular}
\end{center}
\begin{center} % center the table
	\centering
	\begin{tabular}{ |p{3cm}|p{4cm}|p{4cm}|  }
		\hline
		Responsive su Tablet & Il sito deve potersi adattare agli schermi dei seguenti Tablet: 
		\begin{itemize}
			\item Ipad Air, Pro dal 2018 in poi
			\item Tutti i modelli Xiaomi dal 2018 in poi
			\item Tutti i modelli Samsung dal 2018 in poi
		\end{itemize} & \\
		\hline
		
		
	\end{tabular}
\end{center}
\subsection*{RNF7 Mantenibilità e Scalabilità}
% table
\begin{center} % center the table
	\centering
	\begin{tabular}{ |p{3cm}|p{4cm}|p{4cm}|  }
		\hline
		\centering Proprietà & \qquad\quad Descrizione & \qquad\qquad Misura\\ % I found no other way...
		\hline
		Team di manutenzione &  Al sito deve essere affiancato, prima e dopo il rilascio ufficiale, un team di
		manutenzione che si occupi di testare ogni funzionalit`a periodicamente e
		che, su richiesta qualora ci siano problemi, sia pronto a intervenire tempestivamente
		 & \\
		\hline
		sito facilmente mantenibile & Il sito deve possedere le seguenti caratteristiche
		\begin{itemize}
			\item Il codice sorgente del back-end dev'essere modulare, utilizzando un'architetture a microserivizi
			\item Il codice sorgetne deve rispettare le linee guida del linguaggio scelto
		\end{itemize}
		& Conformità Linee guida Javascript
		\\  \hline

	\end{tabular}
\end{center}
\subsection*{RNF8 Conformità}
\begin{center} % center the table
	\centering
	\begin{tabular}{ |p{3cm}|p{4cm}|p{4cm}|  }
		\hline
		\centering Proprietà & \qquad\quad Descrizione & \qquad\qquad Misura\\ % I found no other way...
		\hline
		Conformità leggi & L'applicazione deve essere conforme alle normative di legge in materia di siti web imposti dall'Unione Europea & Conforme \\
		\hline
		Conformità GDPR & L'applicazione deve essere conforme al GDPR, come descritto in RNF3 & Conforme \\
		\hline
		Conformità W3C WAI & L'applicazione deve essere conforme al W3C WAI (Web Accessibility Initiative) & Conforme \\ 
		\hline
	\end{tabular}

\end{center}
\chapter{Analisi del Constesto}

\section{Utenti e Sistemi Esterni}

TODO: Enumerare gli Utenti e Sistemi Esterni


\section{Diagramma di Contesto}

Spiegare la back-end andando su vari livelli di dettaglio:
\begin{itemize}
	\item Context diagram generale
	\item Divisione in processi
	\item Divisione in Sub Processi
	\item Data flow diagram per i processi (e i sub processi se siamo bravi)
\end{itemize}
\chapter{Analisi dei Componenti}

\section{Definizione dei Componenti}

Componenti interni della mia applicazione e come interagiscono\\
Sostanzialmente sono i componenti usati nei RF in questo documento

\section{Diagramma dei Componenti}

\end{document}
\documentclass{report}
\usepackage{float}

% for the image in the title
\usepackage{tikz}

% custom spacing
\usepackage{setspace}
\onehalfspacing

% footer and header
\usepackage{fancyhdr}
% \setlength{\headheight}{15.2pt}

% underlining
\usepackage{ulem}

\usepackage[T1]{fontenc}

% Table of contents link to corresponding sections
\usepackage{hyperref}
\hypersetup{
	colorlinks,
	citecolor=black,
	filecolor=black,
	linkcolor=black,
	urlcolor=black
}

\usepackage{amsmath}
% Remove che "Chapter" string before chapters
\iffalse
\makeatletter
\def\@makechapterhead#1{%
	\vspace*{50\p@}%
	{\parindent \z@ \raggedright \normalfont
		\interlinepenalty\@M
		\Huge\bfseries  \thechapter.\quad #1\par\nobreak
		\vskip 40\p@
}}
\makeatother
\fi

% Fancy chapters
\usepackage[Bjarne]{fncychap}
% options: Sonny, Lenny, Glenn, Conny, Rejne, Bjarne, Bjornstrup

\begin{document}
	
	
	%title page
	\begin{titlepage}
		\begin{figure}[t]
			\centering\includegraphics[width=0.3\textwidth]{images/unitn-logo}
		\end{figure}
		\begin{center}
			\textsc{ \LARGE{Università degli Studi di Trento \\}}
			\textsc{ \LARGE{Facoltà di Informatica\\ }}
			\textnormal{ \LARGE{Corso di Ingegneria del Software\\}}
			\vspace{30mm}
			\fontsize{10mm}{7mm}\selectfont 
			\textup{Fix Mi \\ Report Finale}\\
		\end{center}
		
		\vspace{25mm}
		
		\centering
		\large Gruppo G43: \\ Giovanni Santini\\ Riginel Ungureanu \\ Valerio Asaro
		
		\vspace{20mm}
		
		\centering{\large{Anno Accademico 2023/2024 \\ Trento }}
		
	\end{titlepage}
	
	
	
	
	% use header and footers
	\pagestyle{fancy}
	\fancyhead[R]{\chaptername\ \thechapter}  % header
	
	%\maketitle
	\tableofcontents
	\newpage
	
	
	
	\section{Scopo del documento}
	
	
	\section{Informazioni del Documento}
	
	% table
	\begin{center} % center the table
		\centering
		\begin{tabular}{ |p{4cm}|p{4cm}|  }
			\hline
			\centering Campo & \qquad\qquad Valore \\ % I found no other way...
			\hline
			Titolo del Documento & Report Finale \\
			\hline
			Titolo del Progetto & Fix Mi \\
			\hline
			Autori del Documento &
			Giovanni Santini \\ & Riginel Ungureanu \\ & Valerio Asaro \\
			\hline
			Amministratore Progetto & Riginel Ungureanu\\
			\hline
			Versione del documento & 1.0 \\
			\hline
		\end{tabular}
	\end{center}
	
	
	
\chapter{Report}

\section{Approcci all'ingegneria del software}

\subsection{Blue Tensor}

\subsection{Kanban}

\subsection{IBM}
\subsubsection*{Riassunto}
Il seminario viene  presentato dal Cloud Technical Specialist Ferdinando Gorga di IBM.
Comincia presentandoci i servizi cloud di IBM, i diversi livelli di astrazione dal bare metal al serverless, e i vantaggi della piattaforma.
Procede facendo l'esempio di un'architettura molto vulnerabile, e i vari metodi per migliorarla utilizzando tecnologie cloud.
Successivamente ci mostra il catalogo di IBM cloud, dove è possibile acquistare varie componenti che funzionano tra di loro senza problemi, per poi mostrarci il code engine, che è alla base del serverless. Una particolarità è la load distribution, in quanto le istanze dell'applicazione si moltiplicano e dividono in base al carico,
arrivando perfino a spegnersi in caso di mancanza di traffico. 
Inoltre ci mostra un semplice linguaggio di scripting grafico nominato NodeRED.

Infine ci mostra il concetto di Cloud Satellite: Un modo per integrare i propri computer e server all'interno dell'ecosistema cloud di IBM.
\subsubsection*{Cosa abbiamo imparato}
Abbiamo imparato quanto sia importante il cloud al giorno d'oggi, in quanto i giusti servizio e tecnologie permettono di dedicarsi al codice e non all'infrastruttura sottostante.
\subsection*{Il nostro approccio}
Il deploy non lo abbiamo fatto su nessuna piattaforma di cloud, tuttavia abbiamo seguito l'approccio dei container per i nostri microservizi. 
Grazie a ciò non soffriamo di problemi di compatibilità tra le macchine, il software è portabile e siamo certi che con un po' di lavoro si potrebbe rilasciare sul cloud.

\subsection{META}

\subsection{U-Hopper}

\subsection{RedHat}

\subsection{Microsoft}
\subsubsection*{Riassunto}
Il seminario viene presentato dal Senior Software Engineer Diego Colombo.
Comincia presentandoci il concetto di testing, i suoi utilizzi e i vari tipi, e usa come esempio particolare l'acceptance testing e il behaviour driven development(bdd).
Successivamente ci spiega perchè è importante testare e perchè sono fondamentali degli health check all'interno di un'architettura a microservizi.
Dopodichè dimostra i vari tipi di tool per il testing - dal Unit Test framework ai test runner continui, 
per poi dirci che i test vengono idealmente svolti in diversi contesti e diverse fasi: Nella macchina di sviluppo, nell'environment di build, in production e perfino nel computer dell'utente finale.
Ci mostra degli esempi di strumenti di testing scrivendo il codice in tempo reale:
\begin{itemize}
	\item Mocha, framework di test e Chai, framework di asserzioni 
	\item Wallaby, un test runner continuo e con supporto di live comment
	\item Playwright, che simula un browser per testare le funzionalità di ui e permette di fare mocking
\end{itemize}
Infine ci spiega la filosofia del Test Driven Development, dove i test vanno scritti ancora prima del codice.
\subsubsection*{Cosa abbiamo imparato}
Questo seminario ci ha fatto capire l'importanza dei test e il modo in cui la loro presenza possa facilitare lo sviluppo. Inoltre il test driven development è un approccio interressante che sicuro adotteremo in futuri progetti. 
\subsubsection*{Il nostro approccio}
Per il testing delle nostre API abbiamo usato Jest, molto simile a Mocha, e Supertest, ottima per testare le richieste http. Non abbiamo seguito un approccio TDD, però abbiamo messo grande importanza nel testing di ogni API.

\subsection{Molinari}

\subsection{Marsiglia}
\subsubsection*{Riassunto}
Il seminario viene presentato dal Enterprise Architect Gerardo Marsiglia
Comincia presentandoci i vari ruoli all'interno di un progetto, quali Specialista, Architetto, Consulente, Ingegnere e Manager.
Ci parla del ciclo di vita del software che, soprattutto nei grandi e vari team di sviluppo, è difficile da gestire. E' fondamentale definire standard, organizzare lo sviluppo e utilizzare tool di automazione.
Successivamente ci parla dell'evoluzione delle metodologie di sviluppo, partendo da waterfall, passando a quelle iterative, all'Agile e infine al DevOps.
Dopodichè spiega il concetto di Continous Delivery, processo fondamentale del DevOps che si integra con il Continous Integration. 
Ciò che è più evidente è che il processo di Continous Delivery è spesso automatizzabile, il che velocizza tutto.
Successivamente parla di Modernizzazione Applicativa, un processo fondamentale per modernizzare le applicazioni tradizionali e ottenere i vantaggi delle nuove tecnologie.
In questo segmento parla anche dell'evoluzione dell'architettura software, partendo dal monolith per arrivare poi all'architettura a microservizi, e i vari livelli di astrazione dal Bare metal ai container.
Infine ci mostra l'esempio della domotizzazione di casa sua, dove ha utilizzato una soluzione open source per unire i vari dispositivi di diversi marchi in un' unica applicazione dalla quale controllarli.

\subsubsection*{Cosa abbiamo imparato}
Abbiamo imparato che la gestione del ciclo di vita è veramente importante e non va mai trascurato, ci ha offerto uno scorcio nel mondo del DevOps e in particolare del CI/CD.
\subsubsection*{Il nostro approccio}
Per il progetto non abbiamo seguito una metodologia precisa, tuttavia durante alcune fasi dello sviluppo del codice abbiamo seguito un approccio simile al Continous Integration.
Inoltre abbiamo sviluppato un'architettura a microservizi.
\subsection{APSS \& Trentino.ai}


\section{Organizzazione del lavoro}

\section{Ruoli e Attività}

\section{Carico e distribuzione del lavoro}

\section{Criticità}

\section{Autovalutazione}



	
\end{document}
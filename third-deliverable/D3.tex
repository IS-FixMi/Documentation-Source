\documentclass{report}
\usepackage{float}

% for the image in the title
\usepackage{tikz}

% custom spacing
\usepackage{setspace}
\onehalfspacing

% footer and header
\usepackage{fancyhdr}
% \setlength{\headheight}{15.2pt}

% underlining
\usepackage{ulem}


% Table of contents link to corresponding sections
\usepackage{hyperref}
\hypersetup{
	colorlinks,
	citecolor=black,
	filecolor=black,
	linkcolor=black,
	urlcolor=black
}

\usepackage{amsmath}
% Remove che "Chapter" string before chapters
\iffalse
\makeatletter
\def\@makechapterhead#1{%
	\vspace*{50\p@}%
	{\parindent \z@ \raggedright \normalfont
		\interlinepenalty\@M
		\Huge\bfseries  \thechapter.\quad #1\par\nobreak
		\vskip 40\p@
}}
\makeatother
\fi

% Fancy chapters
\usepackage[Bjarne]{fncychap}
% options: Sonny, Lenny, Glenn, Conny, Rejne, Bjarne, Bjornstrup

\begin{document}
	
	
	%title page
	\begin{titlepage}
		\begin{figure}[t]
			\centering\includegraphics[width=0.3\textwidth]{images/unitn-logo}
		\end{figure}
		\begin{center}
			\textsc{ \LARGE{Università degli Studi di Trento \\}}
			\textsc{ \LARGE{Facoltà di Informatica\\ }}
			\textnormal{ \LARGE{Corso di Ingegneria del Software\\}}
			\vspace{30mm}
			\fontsize{10mm}{7mm}\selectfont 
			\textup{Fix Mi \\ Documento di Architettura}\\
		\end{center}
		
		\vspace{25mm}
		
		\centering
		\large Gruppo G43: \\ Giovanni Santini\\ Riginel Ungureanu \\ Valerio Asaro
		
		\vspace{20mm}
		
		\centering{\large{Anno Accademico 2023/2024 \\ Trento }}
		
	\end{titlepage}
	
	
	
	
	% use header and footers
	\pagestyle{fancy}
	\fancyhead[R]{\chaptername\ \thechapter}  % header
	
	%\maketitle
	\tableofcontents
	\newpage
	
	
	
	\section{Scopo del documento}
	
	Il presente documento a titolo "Documento di Architettura" segue il documento "Specifica dei Requisiti". Il documento descrive il Diagramma delle Classi in "Unified Modeling Language" (UML), in modo continuativo al Diagramma delle Componenti e del Contesto, ai Requisiti Funzionali e Requisiti Non Funzionali, integrato a codice "Object Constraint Language" (OCL). Le classi definite dovranno essere implementate a livello di codice.
	
	
	\section{Informazioni del Documento}
	
	% table
	\begin{center} % center the table
		\centering
		\begin{tabular}{ |p{4cm}|p{4cm}|  }
			\hline
			\centering Campo & \qquad\qquad Valore \\ % I found no other way...
			\hline
			Titolo del Documento & Documento di Architettura \\
			\hline
			Titolo del Progetto & Fix Mi \\
			\hline
			Autori del Documento &
			Giovanni Santini \\ & Riginel Ungureanu \\ & Valerio Asaro \\
			\hline
			Amministratore Progetto & Riginel Ungureanu\\
			\hline
			Versione del documento & 1.0 \\
			\hline
		\end{tabular}
	\end{center}
	


\iffalse
\section{Nome}
\begin{figure}[H]
	\centering\includegraphics[width=1\textwidth]{URL}
	Nome dell'immagine
\end{figure}
\subsection*{Descrizione}
Descrivi generalmente cosa fa l'immagine mostrata

Per ogni classe fai una
\subsubsection{Nome}

in cui spieghi la classe dal punto di vista dell'alto livello, e i metodi (solo quelli non ovvi) mostrati.
In seguito descrivi come interagisce la classe analizzata con le altre.


\fi
	
\chapter{Diagramma delle Classi}

Nel presente capitolo vengono illustrate le classi previste nell'applicazione "FixMi". Tali classi sono rappresentate in "Unified Modeling Language" (UML) e sono caratterizzate da un nome, degli attributi specificando la visibilità ("+" = public, "-" = private) e dei metodi. Ogni metodo è caratterizzato da un valore di ritorno, un nome e degli argomenti. Inoltre alcune classi presentano un tipo al di sopra del nome per identificare proprietà particolari di un classe o aggiungere informazioni, questi sono:
\begin{itemize}
	\item "Enumeration": L'oggetto rappresenta un valore che può essere in uno degli stati descritti all'interno dell'oggetto
	\item "Abstract": Una classe abstract non viene mai istanziata, viene perciò usata per effettuare templating e venir estesa da altre classi.
	\item "Singleton": Una classe che viene istanziata una ed una sola volta
\end{itemize}
Nei diagrammi che seguono vieve rappresentato il rapporto tra diverse classi, utilizzando i simboli definiti sotto.


\section{Definizione Simboli}
\begin{figure}[H]
	\centering\includegraphics[width=1\textwidth]{images/Diagramma_delle_classi_definizioni.png}
	Diagramma delle classi per i simboli 
\end{figure}
Definiamo i seguenti simboli riguardanti la relazione tra più classi:
\begin{itemize}
	\item ISA: vi è un rapporto di ereditarietà tra due classi
	\item Composizione: Stabilisce una relazione per cui una classe è parte di un'altra classe
	\item Aggregazione: Stabilisce una relazione per cui una classe possiede un'altra classe 
\end{itemize}
\section{Tipi di dato}
\begin{figure}[H]
	\centering\includegraphics[width=1\textwidth]{images/Diagramma_delle_classi_Enumi.png}
	Diagramma delle classi per i tipi di dato
\end{figure}
Allo scopo di definire dei tipi di dato chiari e comprensibili, sono state aggiunte le classi ed enumerazioni di cui sotto:
\subsection*{Image}
L'Enumerazione Image rappresenta i diversi formati di immagine utilizzati all'interno dell'applicazione.
\subsection*{PermissionLevel}
L'Enumerazione PermissionLevel rappresenta il livello di permesso di un certo utente
\subsection*{TaskTag}
L'Enumerazione TaskTag rappresenta i vari tipi di TaskTag dell'applicazione
\subsection*{WorkTag}
L'Enumerazione WorkTag rappresenta i vari tipi di WorkTag dell'applicazione
\subsection*{Currency}
L'Enumerazione Currency rappresenta le valute supportate all'interno dell'applicazione
\subsection*{Payment Method}
L'Enumerazione Payment Method rappresenta le piattaforme di pagamento utilizzabili all'interno dell'applicazione 
\subsection*{TaskStatus}
L'Enumerazione TaskStatus rappresenta lo stato di una determinata task.
\subsection*{Date}
La classe Date rappresenta una data
\subsection*{Language}
L'Enumerazione Language rappresenta una lingua supportata dall'applicazione

\section{Microservizio}
\begin{figure}[H]
	\centering\includegraphics[width=1\textwidth]{images/Diagramma_delle_classi_Microservizio.png}
	Diagramma delle classi per i Microservizi
\end{figure}
Data la specifica del Back-End e dato il diagramma di contesto sono state individuate le classi e relativi metodi e attributi di cui sotto
\subsection*{Descrizione}
La classe astratta Microservizio rappresenta un microservizio generico dell'applicazione.
A questi fini sono stati identificati i seguenti attributi e metodi
\begin{itemize}
	\item private mid: Int 
	\begin{itemize}
		\item questo attributo è un codice identificativo univoco per il microservizio.
	\end{itemize}
	\item private indirizzo: IpAddr
	\begin{itemize}
		\item questo attributo corrisponde all'indirizzo IP e porta del microservizio.
	\end{itemize}
	\item public page: html
	\begin{itemize}
		\item questo attributo contiene la pagina principale del microservizio in html 
	\end{itemize}
	
	\item public autenticaUtente(token: Token): PermissionLevel 
	\begin{itemize}
		\item il metodo permette di richiedere al "microservizio autenticazione" il livello di permesso di un determinato utente, che ha fornito il suo token di sessione. ritorna il PermissionLevel dell'utente.
	\end{itemize}
\end{itemize}
\subsection*{Classi figlie}
Le classi figlie di "Microservizio" sono singleton, ciascuna rappresentante uno specifico microservizio dell'applicazione.
Queste sono le seguenti:
\begin{itemize}
	\item Microservizio Task
	\item Autenticazione
	\item Negozio
	\item Home
	\item Magazzino
	\item Gestione Dipendenti
\end{itemize}

\section{Database}
\begin{figure}[H]
	\centering\includegraphics[width=1\textwidth]{images/Diagramma_delle_classi_Database.png}
	Diagramma delle classi per i Database
\end{figure}
Data la specifica del Back-End e il diagramma di Contesto sono state individuate le classi e relativi metodi e attributi di cui sotto:
\subsection*{Descrizione}
La classe astratta Database rappresenta un Database generico dell'applicazione.
A questi fini sono stati identificati i seguenti attributi e metodi
\begin{itemize}
	\item private username: String 
	\begin{itemize}
		\item questo attributo è l'username necessario per accedere al database
	\end{itemize}
	\item private password: Hash 
	\begin{itemize}
		\item questo attributo è l'Hash della password necessaria per accedere al database
	\end{itemize}
	\item public login(username: String, password: Hash): bool 
	\begin{itemize}
		\item questo metodo permette a un utente o a un microservizio di accedere al database per eseguire le proprie query. ritorna true se l'operazione è andata a buon fine, false altrimenti
	\end{itemize}
\end{itemize}
\subsection*{Classi figlie}
Le classi figlie di Database sono singleton, ciascuna rappresenta un specifico database dell'applicazione. 
Questi sono: 
\begin{itemize}
	\item Database Magazzino
	\item Database Autenticazione
	\item Database Task
\end{itemize}
\section{Profilo}
\begin{figure}[H]
	\centering\includegraphics[width=1\textwidth]{images/Diagramma_delle_classi_Profilo.png}
	Diagramma delle classi per i profili
\end{figure}
Data l'analisi del Contesto sezione 1 "Utenti e Sistemi Esterni"  sono state individuate le classi e relativi metodi e attributi di cui sotto:
\subsection*{Descrizione}
La classe astratta Profilo rappresenta il profilo di un utente del sistema che ha eseguito l'autenticazione.
L'utente autenticato nel proprio profilo può utilizzare le funzionalità del sistema attraverso i microservizi, e può visitare le varie pagine in base al proprio PermissionLevel. 
A questi fini sono stati identificati i seguenti metodi e attributi:
\begin{itemize}
	\item private email: String
	\begin{itemize}
		\item questo attributo è l'email appartenente all'utente
	\end{itemize}
	\item private password: Hash 
	\begin{itemize}
		\item questo attributo è l'Hash della password dell'utente
	\end{itemize}
	\item private permissionLevel: PermissionLevel 
	\begin{itemize}
		\item questo attributo è il livello di permesso dell'utente
	\end{itemize}
	\item public goHome(): void 
	\begin{itemize}
		\item questo metodo permette all'utente di andare alla pagina Home
	\end{itemize}
	\item public goNegozio(): void 
	\begin{itemize}
		\item questo metodo permette all'utente di andare alla pagina Negozio
	\end{itemize}
	\item public checkPassword(hash: Hash): bool 
	\begin{itemize}
		\item questo attributo compara l'attributo password con l'hash fornito, ritornando true se combaciano, false altrimenti
	\end{itemize}
\end{itemize}
\subsection*{ProfiloCliente}
La classe ProfiloCliente, figlia della classe "Profilo" rappresenta il profilo di un cliente
Sono stati identificati i seguenti metodi e attributi: 
\begin{itemize}
	\item private token: Token
	\begin{itemize}
		\item questo attributo rappresenta il token di sessione di un cliente
	\end{itemize}
	\item private carrello: Carrello
	\begin{itemize}
		\item questo attributo rappresenta il carrello del cliente.
	\end{itemize}
	\item public new(email: String, password: Hash): ProfiloCliente
	\begin{itemize}
		\item questo metodo permette di creare un nuovo ProfiloCliente contenenti l'email e l'hash password forniti.
	\end{itemize}
	\item public logOut(profilo: ProfiloCliente): bool
	\begin{itemize}
		\item questo metodo permette di eseguire il logOut dall'applicazione, ritorna true se l'operazione è stata eseguita con successo, false altrimenti
	\end{itemize}
\end{itemize}
\subsection*{ProfiloDipendente}
La classe ProfiloDipendente, figlia della classe "ProfiloCliente" rappresenta il profilo di un dipendente dell'azienda.
Sono stati identificati i seguenti metodi e attributi: 
\begin{itemize}
	\item private workTags: WorkTag[]
	\begin{itemize}
		\item questo attributo rappresenta una lista di workTag del dipendente.
	\end{itemize}
	\item private promoteToManager(profiloDipendente): bool
	\begin{itemize}
		\item questo metodo permette di trasformare il profiloDipendente fornito in un profiloManager
	\end{itemize}
\end{itemize}
\subsection*{ProfiloManager}
La classe ProfiloManager, figlia della classe "ProfiloDipendente" rappresenta il profilo del manager dell'azienda.



\section{Autenticazione}

\begin{figure}[H]
	\centering\includegraphics[width=1\textwidth]{images/Diagramma_delle_classi_Autenticazione.png}
	Diagramma delle classi per la componente "Autenticazione" 
\end{figure}

Date le seguenti componenti: 
\begin{itemize}
	\item "Autenticazione"
	\item "Login"
	\item "Registrazione"
	\item "Cambio Password"
\end{itemize}
seguendo ciò che è stato descritto negli RF "1" e "2" sono state individuate le classi e relativi metodi e attributi di cui sotto:
\subsection*{Descrizione}
La classe "Autenticazione" è un singleton che rappresenta il microservizio avente lo stesso nome. 
Questa classe gestisce i dati di autenticazione dei profili quali e-mail, password e token. 
La classe contiene una HashMap che associa a ogni profilo un Token identificativo per la sessione.
A questi fini, sono stati identificati i seguenti metodi:
\begin{itemize}
	\item public login( email: String, password: String ) : bool
		\begin{itemize}
			\item questo metodo permette a un utente di eseguire il login, ritornando true se è avvenuto con successo e false altrimenti
			\item all'interno del metodo, dopo aver eseguito il controllo dell'esistenza di un profilo con stessa email e password,
			viene generato un codice 2FA e inviato alla email rispettiva utilizzando i metodi "generaOTP()" e "sendOTP()", per poi aspettare che l'utente lo inserisca
			\item alla fine del metodo e in caso di esito positivo, un token viene generato , inserito nella session e inviato all'utente utilizzando i metodi generaToken() e sendToken() rispettivamente
		\end{itemize}
	\item public verificaOTP( utente: Profilo, otp: OTP): bool
		\begin{itemize}
			\item questo metodo è necessario per verificare l'identità dell'utente attraverso il codice OTP, ritornando true se l'OTP inserito è giusto.
		\end{itemize}
	\item public sendToken() :bool 
		\begin{itemize}
			\item questo metodo permette di inviare ad un utente il suo token
		\end{itemize}
	\item public registraUtente(email: String, password: String): bool
		\begin{itemize}
			\item questo metodo permette ad un nuovo utente di registrarsi, ritornando true se la registrazione è andata a buon fine
			\item all'interno del metodo, dopo aver controllato che l'email fornita non appartenga a un profilo esistente, viene generato un codice 2FA e inviato alla email rispettiva utilizzando i metodi "generaToken()" e "sendToken()", per poi aspettare che l'utente lo inserisca
		\end{itemize}
	\item public modificaPassword(Profilo, new\_password: Hash ) :bool
		\begin{itemize}
			\item questo metodo permette ad un utente di modificare la propria password, fornendo l'hash della nuova password, e ritorna true se la modifica è andata buon fine, false altrimenti
			\item all'interno del metodo viene generato un codice 2FA e inviato alla email rispettiva utilizzando i metodi "generaToken()" e "sendToken()", per poi aspettare che l'utente lo inserisca
		\end{itemize}
	\item private generaOTP(utente: Profilo) : OTP
		\begin{itemize}
			\item questo metodo genera un codice OTP per un profilo, ritornandolo 
		\end{itemize}
	\item private sendOTP(otp: OTP, email: Email): bool
		\begin{itemize}
			\item questo metodo invia un codice OTP alla email fornita, ritornando true se l'operazione è andata a buon fine, false altrimenti
		\end{itemize}
	\item private generaToken(utente: Profilo) : Token
		\begin{itemize}
			\item questo metodo genera un token per un determinato utente, inserendolo nella session e ritornandolo
		\end{itemize}
	\item private addUtenteDB(utente:Profilo): bool
		\begin{itemize}
			\item questo metodo chiede al DB di aggiungere un nuovo profilo utente 
		\end{itemize}
	\item private getProfiloDB(email: Email) : Profilo
		\begin{itemize}
			\item questo metodo chiede al DB il profilo avente la mail fornita. Se esiste viene ritornato il profilo, altrimenti null
		\end{itemize}
	\item private modificaPasswordDB(profilo: Profilo, new\_password: Hash) : bool
		\begin{itemize}
			\item questo metodo richiede al DB la modifica della password di un dato profilo, e ritorna true se il processo è andato a buon fine, false altrimenti
		\end{itemize}
\end{itemize}
\subsection*{Database Autenticazione}
La classe Database Autenticazione è un singleton che rappresenta il database contenente i profili. 
Permette di gestire i profili, aggiungerli ed eliminarli.
Sono stati individuati i seguenti metodi: 
\begin{itemize}
	\item public getClienti(): ProfiloCliente[]
	\begin{itemize}
		\item questo metodo ritorna tutti i profili cliente immagazinati
	\end{itemize}
	\item public getDipendenti(): ProfiloDipendente[]
	\begin{itemize}
		\item questo metodo ritorna tutti i profili dipendente immagazinati
	\end{itemize}
	\item public getProfilo(email: String): Profilo
	\begin{itemize}
		\item questo metodo ritorna il profilo a cui corrisponde l'email fornita. Se non esiste ritorna null
	\end{itemize}
	\item public getCliente(email: String): ProfiloCliente
	\begin{itemize}
		\item questo metodo ritorna il profilo cliente a cui corrisponde l'email fornita. se non esiste ritorna null
	\end{itemize}
	\item public getDipendente(email String): ProfiloDipendente
	\begin{itemize}
		\item questo metodo ritorna il profilo dipendente a cui corrisponde l'email fornita. se non esiste ritorna null
	\end{itemize}
	\item public addUtente(profilo: Profilo): bool
	\begin{itemize}
		\item questo metodo inserisce il profilo fornito nel database. ritorna true se l'operazione è avvenuta con successo, false altrimenti
	\end{itemize}
	\item public eliminaUtente(profilo: Profilo): bool
	\begin{itemize}
		\item questo metodo elimina il profilo fornito dal database. ritorna true se l'operazione è avvenuta con successo, false altrimenti
	\end{itemize}
	\item public modificaPassword(profilo: Profilo, new\_password: Hash): bool
	\begin{itemize}
		\item questo metodo modifica la password immagazzinata nel database del profilo fornito. ritorna true se l'operazione è avvenuta con successo, false altrimenti
	\end{itemize}
\end{itemize}
\subsection*{Email Server}
La classe Email Server è un singleton che rappresenta il server email utilizzato dall'azienda. permette di inviare email.
Sono stati individuati i seguenti metodi:
\begin{itemize}
	\item public sendEmail(receiver: Email, sender: Email,content: String) : bool
	\begin{itemize}
		\item questo metodo invia una mail dall'email del mittente(sender) all'email del destinatario(receiver) con il contenuto(content)
	\end{itemize}
\end{itemize}
\subsection*{Profilo}
La classe astratta Profilo rappresenta il profilo di un determinato utente. Viene gestito dal microservizio Autenticazione e immagazzinato nel database Autenticazione.
Sono stati individuati i seguenti metodi:
\begin{itemize}
	\item public login(email: String, password: Hash) : bool
	\begin{itemize}
		\item questo metodo permette a un utente di svolgere il login nel proprio profilo, fornendo l'email e la password. restituisce true se l'operazione è andata a buon fine, false altrimenti
	\end{itemize}
\end{itemize}











\section{Task}

\begin{figure}[H]
	\centering\includegraphics[width=1\textwidth]{images/Diagramma_delle_classi_task.png}
	Diagramma delle classi per la componente "Task"
\end{figure}
Data la componente "Task", seguendo ciò che è stato descritto nel Requisito Funzionale numero "9" e definita precedentemente nel rispettivo diagramma, sono state individuate le classi e relativi metodi e attributi di cui sotto:
\subsection*{Classe Task}
\begin{figure}[H]
	\centering\includegraphics[width=1\textwidth]{images/Diagramma_delle_classi_task1.png}
	Diagramma delle classi per la componente "Task", 1
\end{figure}
\subsection*{Descrizione}
La classe "Task" definisce, come descritto dal Requisito Funzionale numero "9.1", l'oggetto "Task", da cui "TaskNegozio", "TaskMagazzino" e "TaskSupport" ereditano le proprietà. Le variabili di tipo "Task" vengono poi utilizzate dalla classe "Microservizio Task".

\subsection*{TaskRiparazione}
Eredita dalla classe "Task", si specializza nella gestione delle Task per la Riparazione. Dato il Requisito Funzionale 9.1, sono stati individuati i seguenti metodi:

\begin{itemize}
\item private nomeRichiedente: String
\begin{itemize}
	\item Una stringa contenente un nome.
\end{itemize}
\item private cognomeRichiedente: String
\begin{itemize}
	\item Una stringa contenente un cognome.
\end{itemize}
\item private emailRichiedente: String
\begin{itemize}
	\item Una stringa contenente un indirizzo email.
\end{itemize}
\item private image: Image[]
\begin{itemize}
	\item Contiene un gruppo di immagini.
\end{itemize}
\item private phoneNumber: String
\begin{itemize}
	\item Una stringa contenente un numero di telefono.
\end{itemize}
\item public getNomeRichiedente(): String
\begin{itemize}
	\item Ritorna una stringa contenente il nome del richiedente.
\end{itemize}
\item public getCognomeRichiedente(): String
\begin{itemize}
	\item Ritorna una stringa contenente il cognome del richiedente.
\end{itemize}
\item public getImages(): Image[]
\begin{itemize}
	\item Ritorna una stringa contenente le immagini della richiesta di riparazione.
\end{itemize}
\item public getEmail(): String
\begin{itemize}
	\item Ritorna una stringa contenente l'email del richiedente.
\end{itemize}
\item public getPhoneNumber(): String
\begin{itemize}
	\item Ritorna una stringa contenente il numero di telefono del richiedente.
\end{itemize}
\item private setNomeRichiedente(nome: String): bool
\begin{itemize}
	\item Imposta il nome del richiedente, ritorna un valore booleano per indicare il successo dell'operazione.
\end{itemize}
\item private setCognomeRichiedente(cognome: String): bool
\begin{itemize}
	\item Imposta il cognome del richiedente, ritorna un valore booleano per indicare il successo dell'operazione.
\end{itemize}
\item private setImages(images: Image[]): bool
\begin{itemize}
	\item Imposta le immagini per la richiesta di riparazione, ritorna un valore booleano per indicare il successo dell'operazione.
\end{itemize}
\item private setEmail(email: String): bool
\begin{itemize}
	\item Imposta l'email del richiedente, ritorna un valore booleano per indicare il successo dell'operazione.
\end{itemize}
\item private setPhoneNumber(phoneNumber: String): bool
\begin{itemize}
	\item Imposta il numero di telefono del richiedente, ritorna un valore booleano per indicare il successo dell'operazione.
\end{itemize}
\end{itemize}



\subsection*{TaskAssistenza}
Eredita dalla classe "Task", si specializza nella gestione delle Task per la Assistenza. Dato il Requisito Funzionale 9.1, sono stati individuati i seguenti metodi:
\begin{itemize}
\item private emailRichiedente: String
\begin{itemize}
	\item Una stringa contenente un indirizzo email.
\end{itemize}
\item public getEmailRichiedente(): String
\begin{itemize}
	\item Ritorna una stringa contenente l'email del richiedente.
\end{itemize}
\item private  setEmailRichiedente(email: String): bool
\begin{itemize}
	\item Imposta l'indirizzo email del richiedente, ritorna un valore booleano per indicare il successo dell'operazione.
\end{itemize}
\end{itemize}


\subsection*{TaskFeedback}
Eredita dalla classe "Task", si specializza nella gestione delle Task per il Feedback. Dato il Requisito Funzionale 9.1, sono stati individuati i seguenti metodi:
\begin{itemize}

\item private ideePerMigliorare: String
\begin{itemize}
	\item Una stringa contenente le idee per migliorare.
\end{itemize}
\item private disponibilitàAzienda: String
\begin{itemize}
	\item Una stringa contenente la disponibilità dell'azienda.
\end{itemize}
\item private velocitàRiparazione: String
\begin{itemize}
	\item Una stringa contenente la velocità della riparazione.
\end{itemize}
\item private soddisfazioneRiparazione: String
\begin{itemize}
	\item Una stringa contenente la soddisfazione della riparazione.
\end{itemize}
\item private soddisfazioneSitoWeb: String
\begin{itemize}
	\item Una stringa contenente la soddisfazione del sito Web.
\end{itemize}

\item public getIdeePerMigliorare(): String
\begin{itemize}
	\item Ritorna una stringa contenente le idee per migliorare.
\end{itemize}
\item public getDisponibilitàAzienda(): String
\begin{itemize}
	\item Ritorna una stringa contenente la disponibilità dell'azienda.
\end{itemize}
\item public getVelocitàRiparazione(): String
\begin{itemize}
	\item Ritorna una stringa contenente la velocità dalla riparazione.
\end{itemize}
\item public getSoddisfazioneRiparazione(): String
\begin{itemize}
	\item Ritorna una stringa contenente la soddisfazione della riparazione.
\end{itemize}
\item public getEmail(): String
\begin{itemize}
	\item Ritorna una stringa contenente la soddisfazione del sito web.
\end{itemize}
\item private setIdeePerMigliorare(idea: String): bool
\begin{itemize}
	\item Imposta una stringa contenente le idee per migliorare il sito, ritorna un valore booleano per indicare il successo dell'operazione.
\end{itemize}
\item private setDisponibilitàAzienda(disponibilità: String): bool
\begin{itemize}
	\item \item Imposta una stringa contenente le disponibilità dell'azienda, ritorna un valore booleano per indicare il successo dell'operazione.
\end{itemize}
\item private setVelocitàRiparazione(velocità: String): bool
\begin{itemize}
	\item Imposta una stringa contenente la velocità della riparazione, ritorna un valore booleano per indicare il successo dell'operazione.
\end{itemize}
\item private setSoddisfazioneRiparazione(soddisfazione: String): bool
\begin{itemize}
	\item Imposta una stringa contenente la soddisfazione della riparazione, ritorna un valore booleano per indicare il successo dell'operazione.
\end{itemize}
\item private setSoddisfazioneSitoWeb(soddisfazione: String): bool
\begin{itemize}
	\item Imposta una stringa contenente la soddisfazione del sito Web, ritorna un valore booleano per indicare il successo dell'operazione.
\end{itemize}
\end{itemize}





\begin{figure}[H]
	\centering\includegraphics[width=1.2\textwidth]{images/Diagramma_delle_classi_task2.png}
	Diagramma delle classi per la componente "Task", 2
\end{figure}
\subsection{Microservizio Task}
\subsection*{Descrizione}

\subsubsection*{Microservizio Task}
Si pone come tramite tra il dipendente ed il Database Task. Invia e riceve informazioni riguardanti le Task dal Database, permette al dipendente di creare una task, rimuoverla o sceglierla (Cambiandone lo stato in "In Lavorazione"). Per tale scopo, sono stati individuati i seguenti metodi ed attributi:

\begin{itemize}
\item public creaTask(args):bool
\begin{itemize}
    \item Crea un nuovo oggetto di tipo "Task", ritorna un valore booleano per indicare il successo dell'operazione. Gli argomenti contengono i valori dei campi appartenenti alla classe "Task". 
\end{itemize}
\item public rimuoviTask(task: Task): bool
\begin{itemize}
    \item Rimuove un oggetto di tipo "Task", ritorna un valore booleano per indicare il successo dell'operazione. L'argomento è l'oggetto "Task" da eliminare.
\end{itemize}
\item public scegliTask(dipendente: ProfiloDipendente, task: Task): bool
\begin{itemize}
    \item Sceglie un oggetto di tipo "Task" da assegnare ad un dipendente, fallisce nel caso in cui un dipendente stia ancora lavorando ad un altra task.
\end{itemize}
\item public modificaStatoTask(task: Task, stato: TaskStatus): bool
\begin{itemize}
    \item Modifica lo stato di un oggetto di tipo "Task", ritorna un valore booleano per indicare il successo dell'operazione.
\end{itemize}
\item public getListaTaskInLavorazione(): Task[]
\begin{itemize}
    \item Ritorna una lista di tipo "Task" contenente oggetti "Task" che sono ancora "In Lavorazione".
\end{itemize}
\item public getListaTaskInPausa(): Task[]
\begin{itemize}
    \item Ritorna una lista di tipo "Task" contenente oggetti "Task" che sono "In Pausa".
\end{itemize}
\item public getStoricoTask(dipendente: ProfiloDipendente): Task[]
\begin{itemize}
    \item Ritorna una lista di tipo "Task" contenente oggetti "Task" completati da un dato dipendente.
\end{itemize}
\item private creaTaskDB(args):bool
\begin{itemize}
    \item Salva i dati di un oggetto di tipo "Task" all'interno del Database, ritorna un valore booleano per indicare il successo dell'operazione.
\end{itemize}
\item private rimuoviTaskDB(task: Task): bool
\begin{itemize}
    \item Rimuove i dati di un oggetto di tipo "Task" dal Database, ritorna un valore booleano per indicare il successo dell'operazione.
\end{itemize}
\item private getListaTaskInDB(String query): Task[]
\begin{itemize}
    \item Ritorna una lista di tipo "Task" contenente il risultato  di una richiesta effettuata sul Database.
\end{itemize}
\item private getStoricoTaskDipendenteDB(dipendente: ProfiloDipendente): Task[]
\begin{itemize}
    \item Ritorna una lista di tipo "Task" contenente lo storico delle task svolte da un dipendente, effettuando una richiesta al DataBase.
\end{itemize}
\end{itemize}

\subsubsection*{Database Task}
La classe "Database Task" permette al Microservizio Task di interfacciarsi con il Database, fornisce metodi utili alla memorizzazione delle tasks, alla loro gestione ed al loro recupero. I metodi individuati sono i seguenti:
\begin{itemize}
\item task: Task[]
\begin{itemize}
	\item L'insieme di oggetti "Task" memorizzato in maniera persistente.
\end{itemize}
\item public addTask(Task): bool
\begin{itemize}
    \item Aggiunge un nuovo elemento di tipo "Task" al Database, ritorna un valore booleano per indicare il successo dell'operazione.
\end{itemize}
\item public rimuoviTask(Task): bool
\begin{itemize}
    \item Rimuove un elemento esistente di tipo "Task" dal Database, ritorna un valore booleano per indicare il successo dell'operazione.
\end{itemize}
\item public listaTask(String query): Task[]
\begin{itemize}
    \item Ritorna una lista di tipo "Task", risultato della ricerca fatta all'interno del Database tramite la stringa "query".
\end{itemize}
\item public storicoTask(ProfiloDipendente): Task[]
\begin{itemize}
    \item Ritorna una lista di tipo "Task" contenente tutte le task completate dal profilo dipendente richiesto.
\end{itemize}
\end{itemize}

\subsubsection*{Profilo Dipendente}
Il dipendente può accedere alla pagina "Tasks"( in cui le può visualizzare, gestire, creare ed eliminare) attraverso il seguente metodo:

\begin{itemize}
\item public goTasks(): void
\begin{itemize}
    \item Porta il dipendente nella pagina per gestire le tasks.
\end{itemize}
\iffalse
METODI PER IL PROFILO
\item public new(args): ProfiloDipendente
\begin{itemize}
    \item Crea e restituisce un nuovo profilo dipendente passando tutte le informazioni necessarie quali "nome", "cognome", "data di nascita" e "giorno di assunzione" come argomenti.
\end{itemize}
\item public getName(): String
\begin{itemize}
    \item Ritorna il nome del profilo dipendente.
\end{itemize}
\item public getSurname(): String
\begin{itemize}
    \item Ritorna il cognome del profilo dipendente.
\end{itemize}
\item public getDataDiNascita(): Date
\begin{itemize}
    \item Ritorna il giorno di nascita del profilo dipendente.
\end{itemize}
\item public getGiornoAssunzione(): Date
\begin{itemize}
    \item Ritorna il giorno di assunzione del profilo dipendente.
\end{itemize}
\item private setName(nome: String): bool
\begin{itemize}
    \item Imposta il nome del profilo dipendente, ritorna un valore booleano per indicare il successo dell'operazione.
\end{itemize}
\item private setSurname(cognome: String): bool
\begin{itemize}
    \item Imposta il cognome del profilo dipendente, ritorna un valore booleano per indicare il successo dell'operazione.
\end{itemize}
\item private  setDataDiNascita(dataDiNascita: Date): bool
\begin{itemize}
    \item Imposta la data di nascita del profilo dipendente, ritorna un valore booleano per indicare il successo dell'operazione.
\end{itemize}
\item private setGiornoDiAssunzione(giornoAssunzione: Date): bool
\begin{itemize}
    \item Imposta il giorno di assunzione del profilo dipendente, ritorna un valore booleano per indicare il successo dell'operazione.
\end{itemize}
\item private promoteToManager(ProfiloDipendente): bool
\begin{itemize}
    \item Trasforma un profilo dipendente in un profilo manager, ritorna un valore boolean per indicare il successo dell'operazione.
\end{itemize}
FINE METODI PER IL PROFILO
\fi
\iffalse
METODI NON CERTI
\item public getWorkTags(): WorkTag[]
\begin{itemize}
    \item Riceve le Work Tags del profilo dipendente.
\end{itemize}
\item public getCurrTask(): Task
\begin{itemize}
    \item Ritorna un oggetto di tipo "Task", che rappresenta la task a cui il dipendente sta lavorando in quel momento.
\end{itemize}
\item public setWorkTags(tags: WorkTag[]): bool
\begin{itemize}
    \item Imposta le Work Tags del profilo dipendente, ritorna un valore booleano per indicare il successo dell'operazione.
\end{itemize}
\item public setCurrTask(task: Task): bool
\begin{itemize}
    \item Imposta un oggetto di tipo "Task" come task a cui il dipendente sta lavorando in quel momento, ritorna un valore booleano per indicare il successo dell'operazione.
\end{itemize}
FINE METODI NON CERTI
\fi
\end{itemize}

\section{Home}
\begin{figure}[H]
	\centering\includegraphics[width=1\textwidth]{images/Diagramma_delle_classi_home.png}
	Diagramma delle classi per "Home"
\end{figure}
Svolge la funzione di accogliere i visitatori nel sito "Fix Mi" e di permettere loro di accedere ad alcuni dei servizi principali quali:
\begin{itemize}
	\item Richiedere assistenza al personale.
	\item Richiedere una riparazione del proprio prodotto ai tecnici.
	\item Inviare feedback riguardo alla qualità dei servizi offerti dallo staff e dal sito web.
\end{itemize}
\subsection*{Home}
\subsection*{Descrizione}
Rappresenta la prima pagina mostrata all'utente o chi, per la prima volta, visita il sito "Fix Mi". Da qui, è possibile raggiungere tutti gli altri servizi attraverso i metodi descritti sotto:
\begin{itemize}
	\item public goRiparazione(): void
	\begin{itemize}
		\item Manda il browser alla pagina "Riparazione".
	\end{itemize}
	\item public goAssistenza(): void
	\begin{itemize}
		\item Manda il browser alla pagina "Assistenza".
	\end{itemize}
	\item public goFeedback(): void
	\begin{itemize}
		\item Manda il browser alla pagina "Feedback".
	\end{itemize}
\end{itemize}

\subsection*{Riparazione}
\subsection*{Descrizione}
Rappresenta la pagina in cui è possibile richiedere una riparazione dai tecnici "Fix Mi". I metodi individuati sono:
\begin{itemize}
	\item public inviaRiparazione(args): void
	\begin{itemize}
		\item Manda la richiesta di riparazione compilata dall'utente al sistema.
	\end{itemize}
	\item private creaTaskRiparazione(args): void
	\begin{itemize}
		\item Crea una nuova task di tipo "Riparazione", contiene le informazioni specifiche alla riparazione stessa passategli per argomenti.
	\end{itemize}
	\item public getListaRiparazioni(utente: Profilo): TaskRiparazione[]
	\begin{itemize}
		\item Ritorna la lista delle richieste di riparazioni già effettuate da un dato utente passato come argomento al metodo.
	\end{itemize}
\end{itemize}

\subsection*{Assistenza}
\subsection*{Descrizione}
Rappresenta la pagina in cui è possibile richiedere una assistenza dai tecnici "Fix Mi". I metodi sono:
\begin{itemize}
	\item public inviaRiparazione(args): void
	\begin{itemize}
		\item Manda la richiesta di assistenza compilata dall'utente al sistema.
	\end{itemize}
	\item private creaTaskAssistenza(args): void
	\begin{itemize}
		\item Crea una nuova task di tipo "Assistenza" contenente le informazioni specifiche alla assistenza passando le informazioni necessarie per argomento.
	\end{itemize}
	%%GETLISTA ASSISTENZA?????
\end{itemize}

\subsection*{Feedback}
\subsection*{Descrizione}
Rappresenta la pagina in cui è possibile mandare un "feedback" riguardo al servizio "Fix Mi". I metodi individuati che verranno utilizzati sono:
\begin{itemize}
	\item public inviaFeedback(): void
	\begin{itemize}
		\item Manda il feedback compilato dall'utente al sistema.
	\end{itemize}
	\item private creaTaskFeedback(args): void
	\begin{itemize}
		\item Crea una nuova task di tipo "Feedback", contiene le informazioni specifiche al feedback stesso tramite argomenti passati al metodo.
	\end{itemize}
	\item private getListaFeeback(): TaskFeedback[]
	\begin{itemize}
		\item Ritorna la lista delle task feedback presenti nel sistema.
	\end{itemize}
\end{itemize}


\section{Carrello}
\begin{figure}[H]
	\centering\includegraphics[width=1\textwidth]{images/Diagramma_delle_classi_carrello.png}
	Diagramma delle classi per la componente "Carrello", mostrando le classi "Carrello", "Item", "Profilo Cliente" e "Database magazzino".
\end{figure}

\subsection*{Descrizione}
Data la componente "Carrello", definita nei documenti precedenti, in riferimento ai Requisiti Funzionali numero "5.1" e "5.2", sono state individuate le classi "Carrello" e "Item". La classe Carrello contiene una lista di "Items" ed offre una serie di funzionalità come l'elencazione, l'aggiunta e la rimozione di Items dalla lista. Inoltre, ogni "Profilo Cliente", definito in precedenza, è composto anche da uno ed uno solo Carrello. A questi fini, sono stati individuati i seguenti metodi per la classe Carrello:

\begin{itemize}
\item addItem(Item[]): bool
\begin{itemize}
	\item Questo metodo permette di aggiungere uno o più Items al Database. Ritorna true se l'operazione termina con successo senza error, altrimenti ritorna flase.
\end{itemize}

\item getSize(): int
\begin{itemize}
	\item Questo metodo ritorna il numero di Items presenti nel carrello.
\end{itemize}

\item getItems(): Item[]
\begin{itemize}
	\item Questo metodo ritorna la lista di Item presente nel carrello.
\end{itemize}

\item isEmpty(): bool
\begin{itemize}
	\item Questo metodo ritorna true se non ci sono elementi nel carrello, altrimenti ritorna false.
\end{itemize}

\item removeItem(Item): bool
\begin{itemize}
	\item Dato un Item come parametro, questo metodo rimuove lo stesso dalla lista degli Items nel carrello, ritornando il risultato dell'operazione.
\end{itemize}

\item removeAll(): bool
\begin{itemize}
	\item Questo metodo rimuove tutti gli elementi dal carrello.
\end{itemize}


\end{itemize}


\subsection*{Item}
Un Item è un'oggetto presente nel magazzino, che può essere acquistato. Sono stati individuati i seguenti metodi e attributi:

\begin{itemize}

\item name: String
\begin{itemize}
	\item Il nome dell'Item.
\end{itemize}

\item picture: Image[]
\begin{itemize}
	\item Una lista di immagini associate all'Item.
\end{itemize}

\item description: String
\begin{itemize}
	\item Una descrizione dell'Item
\end{itemize}

\item prezzo: (double, Currency)
\begin{itemize}
	\item Il prezzo, salvato come una coppia di un numero reale (double) e una Currency
\end{itemize}

\item  itemId: int
\begin{itemize}
	\item Un numero identificativo ed univoco dell'Item
\end{itemize}

\item  new(args): Item
\begin{itemize}
	\item Questo metodo svolge la funzione di costruttore dell'Item, gli attributi vengono passati come argomenti alla funzione stessa. Ritorna un nuovo oggetto Item con i alori passati.
\end{itemize}

\item  delete(): bool
\begin{itemize}
	\item Questo metodo permette di eliminare l'istanza dell'oggetto Item.
\end{itemize}

\end{itemize}

Segue un'elencazione di metodi getter e setter degli attributi specificati sopra:
\begin{itemize}
\item getName(): String

\item getPicture(): Picture

\item getDescription(): String

\item getPrice(): (double, Money)

\item getItemId(): int

\item setName(name: String): bool

\item setPictures(picutres: Picture[]): bool

\item setDescription(desc: String): bool

\item setPrice(cost: double, currency: Currency): bool

\end{itemize}

\subsection*{ProfiloCliente}

Ogni profilo cliente è composto da uno e uno solo carrello, inizialmente inizializzato con una lista vuota di Items. Il Cliente potrà accedere ed interagire con il proprio carrello nella pagina negozio speficata successivamente.


\section{Negozio}


\begin{figure}[H]
	\centering\includegraphics[width=1\textwidth]{images/Diagramma_delle_classi_negozio_2.png}
	Diagramma delle classi per la componente "Negozio" in relazione alle classi "Database Magazzino" e "Profilo"
\end{figure}

\subsection*{Descrizione}

Data la componente "Negozio", definita nel diagramma delle componenti, è stata individuata la classe "Negozio". Tale classe si pone come tramite tra l'utente e il Database Magazzino, dando la possibilità di visionare gli articoli presenti nel magazzino e, qualora l'utente fosse autenticato, di aggiungere / rimuovere articoli dal carrello. A tali fini, in riferimento alla componente "Negozio" citata sopra, vengono fortini  i seguenti metodi pubblici:

\begin{itemize}

\item mostraArticoli(String query): Item[]
\begin{itemize}
	\item Questo metodo ritorna una lista di Item data una query al DB. 
\end{itemize}

\item viewArticolo(Item): bool
\begin{itemize}
	\item Questo metodo permette di visualizzare le informazioni relative ad un'articolo nella pagina del negozio. Ritorna true se l'operazione avviene con successo, altrimenti false.
\end{itemize}


\item getCarrello(ProfiloCliente): Carrello
\begin{itemize}
	\item Dato un ProfiloCliente, questo metodo ritorna il carrello associato al profilo.
\end{itemize}


\item aggiungiAlCarrello(Item, Carrello): bool
\begin{itemize}
	\item Questo metodo permette di aggiungere un item al carrello, dati questi elementi, interfacciandosi al databse. Ritorna true se il procedimento è andato a buon fine, false altrimenti.
\end{itemize}


\item acquista(Carrello): bool
\begin{itemize}
	\item Questo metodo permette di acquistare gli item inseriti nel carrello. Ritorna true se il procedimento è andato a buon fine, false altrimenti.
\end{itemize}

\end{itemize}

Questi metodi si interfacciano alla classe Database Magazzino utilizzando i seguenti metodi privati:
\begin{itemize}
	\item getArticoliDB(String query): Item[]
	\item aggiungiAlCarrelloDB(Item, Carrello): bool
	\item getCarrelloDB(ProfiloCliente): bool
\end{itemize}

\subsection*{Database Magazzino}

La classe Database Magazzino ha il compito di interfacciarsi direttamente con il Database per salvare in modo persistente gli articoli e i carrelli. la classe presenta dei metodi pubblici per interagire con la base di dati come setters e getters. 

In particolare sono stati individuati i seguenti metodi:


\begin{itemize}
	\item cercaArticoli(Strgin query): Item[]
	\begin{itemize}
		\item Questo metodo permette di interrogare il database tramite una query, ritornando il risultato dell'interrogazione. 
	\end{itemize}
	\item rimuoviArticolo(Item): bool
	\begin{itemize}
		\item Questo metodo permette di rimuovere un'articolo dal database. Ritorna true solo se l'operazione è terminata con successo, false altrimenti.
	\end{itemize}
	\item aggiungiArticolo(Item): bool
	\begin{itemize}
		\item Questo metodo permette di aggiungere un'articolo ad database. Ritorna true solo se l'operazione è terminata con successo, altrimenti false.
	\end{itemize}
	\item getCarrello(profilo: ProfiloCliente): Carrello
	\begin{itemize}
		\item Questo metodo ritorna il carrello associato ad un ProfiloCliente.
	\end{itemize}
	\item aggiungiCarrello(Carrello, Item): bool
	\begin{itemize}
		\item Questo metodo permette di aggiungere un item al carrello. Richiede come parametri questi stessi e ritorna true se l'operazione è avvenuta con successo, altrimenti false.
	\end{itemize}
	
\end{itemize}

\subsection*{Profilo}
Ogni utente è in grado di accedere al Negozio, dunque è stato individuato il metodo "goNegozio()" che permetterà all'utente anche non autenticato di accedere alla pagina Negozio.

Il seguente diagramma riassume quanto detto precedentemente riguardo le classi Negozio, Database Magazzino, Profilo Cliente, Carrello, Item.


\begin{figure}[H]
	\centering\includegraphics[width=1\textwidth]{images/Diagramma_delle_classi_negozio_1.png}
	Diagramma delle classi per la componente "Negozio" in relazione alle classi "Database Magazzino" e "Profilo Cliente"
\end{figure}



\section{Magazzino}

\begin{figure}[H]
	\centering\includegraphics[width=1\textwidth]{images/Diagramma_delle_classi_magazzino.png}
	Diagramma delle classi per la componente "Magazzino", mostrando le classi "Magazzino", "ProfiloDipendente" e "Database Magazzino".
\end{figure}

\subsection*{Descrizione}
Data la componente Magazzino, secondo quanto definito in Requisito Funzionale numero "10", viene individuata la classe Magazzino. Tale classe è composta da una pagina Magazzino e permette al dipendente di interagire con il Database Magazzino, aggiungendo e rimuovendo Articoli dall stesso e dando la possibilità di cercare tali articoli attraverso interrogazioni al database. 

Vengono definiti i seguenti metodi:


\begin{itemize}
\item mostraArticoli(): Item[]
\begin{itemize}
	\item Questo metodo ritorna tutti gli articoli presenti nel Database Magazzino.
\end{itemize}

\item cercaArticoli(query: String): Item[]
\begin{itemize}
	\item Questo metodo permette di interrogare il Database Magazzino passando come argomento una stringa "query" che contiene la richiesta.
\end{itemize}

\item aggiungiArticolo(Item): bool
\begin{itemize}
	\item Questo metodo permette di aggiungere un oggetto Item al Database Magazzino.
\end{itemize}

\item rimuoviArticolo(Item): bool
\begin{itemize}
	\item Questo metodo permette al dipendente di rimuovere un oggetto Item dal Database Magazzino.
\end{itemize}

\end{itemize}

La classe Magazzino interagisce con il Database Magazzino attraverso i metodi:

\begin{itemize}
\item cercaArticoliDB(String query): item[]
\item rimuoviArticoloDB(Item): bool
\item aggiungiArticoloDB(Item): bool
\end{itemize}


\subsection*{ProfiloDipendente}

Il profilo dipendente è in grado di raggiungere la pagina magazzino attraverso il metodo "goMagazzino()".

\subsection*{Database Magazzino}

La classe magazzino interagisce direttamente con la classe Database Magazzino che garantisce la persistenza dei dati.







\chapter{Codice in Object Constraint Language}	
In questo capitolo è descritta in modo formale la logica prevista nell’ambito di alcune operazioni di alcune classi.
Tale logica viene descritta in Object Constraint Language (OCL) perché tali concetti non sono esprimibili in nessun altro modo formale nel contesto di UML.	
\section{Microservizio Task}
\begin{figure}[H]
	\centering\includegraphics[width=1\textwidth]{images/OCL/OCL_task.png}
	Diagramma delle classi con linguaggio OCL 
\end{figure}
Quando un dipendente sceglie una task su cui lavorare non deve avere nessuna task correntemente assegnata.
\begin{verbatim}
	context: Microservizio Task::scegliTask(dipendente, task)
	pre: dipendente.getCurrTask() = null
	post: dipendente.currTask = task

\end{verbatim}

\section{Carrello}
\begin{figure}[H]
	\centering\includegraphics[width=1\textwidth]{images/OCL/OCL_carrello.png}
	Diagramma delle classi con linguaggio OCL 
\end{figure}
Per rimuovere un articolo da un carrello è necessario che quell'articolo faccia parte del carrello.
\begin{verbatim}
	context: Carrello::removeItem(item)
	pre: carrello.items->contains(item)

\end{verbatim}
Per comprare gli articoli del carrello è necessario che il carrello non sia vuoto
\begin{verbatim}
	context: Carrello::buy()
	pre: carrello.items->notEmpty()

\end{verbatim}

\section{Autenticazione}
\begin{figure}[H]
	\centering\includegraphics[width=1\textwidth]{images/OCL/OCL_Autenticazione.PNG}
	Diagramma delle classi con linguaggio OCL 
\end{figure}
una volta generato un Token utente, esso viene automaticamente inserito dentro alla session
\begin{verbatim}
	context: Autenticazione::generaToken(utente) returns token
	post: Autenticazione.session[token] = utente

\end{verbatim}
Una volta modificata la password di un profilo utente, allora la password contenuta all'interno dello stesso dev'essere quella nuova
\begin{verbatim}
	context: Autenticazione::modificaPassword(profilo,new_password) returns boolean
post: if (boolean) then profilo.password = new_password

\end{verbatim}
Una volta registrato un utente, il suo profilo deve apparire nel database
\begin{verbatim}
	context: Autenticazione::registraUtente(email,password) returns boolean
	let profilo = profilo creato da registraUtente
	post: if (boolean) then DatabaseAutenticazione.profili->contains(profilo)

\end{verbatim}
\section{Gestione Dipendenti}
\begin{figure}[H]
	\centering\includegraphics[width=1\textwidth]{images/OCL/OCL_Gestione_Dipendenti.PNG}
	Diagramma delle classi con linguaggio OCL 
\end{figure}
Una volta che un profilo dipendente viene aggiunto, deve apparire all'interno del database
\begin{verbatim}
	context: addDipendente(profiloDipendente) returns boolean
	post:if (boolean) then DatabaseAutenticazione.profili->contains(profiloDipendente)
\end{verbatim}
Una volta che un profilo dipendente viene rimosso, non deve più apparire all'interno del database
\begin{verbatim}
	context: removeDipendente(profiloDipendente) returns boolean
	post:if (boolean) then NOT DatabaseAutenticazione.profili->contains(profiloDipendente)

\end{verbatim}

\section{Negozio}
\begin{figure}[H]
	\centering\includegraphics[width=1\textwidth]{images/OCL/OCL_negozio.png}
	Diagramma delle classi con linguaggio OCL 
\end{figure}
Una volta acquistati gli articoli, il database non deve più contenerli. Inoltre dev'essere creata una task magazzino.
\begin{verbatim}
	context: Negozio::acquista(carrello) returns boolean
	post: if (boolean) then ( NOT DatabaseMagazzino.articoli->contains(carrello.items))
	AND
	let t : task magazzino generata 
	post: if (boolean) then(DatabaseTask.tasks->contains(t) )
	
\end{verbatim}
\section{Assistenza, Riparazione, Feedback}
\begin{figure}[H]
	\centering\includegraphics[width=1\textwidth]{images/OCL/OCL_assistenza_riparazione_feedback.PNG}
	Diagramma delle classi con linguaggio OCL 
\end{figure}
Una volta inviata una richiesta di assistenza, riparazione o feedback, viene creata la corrispettiva task e inserita nel database
\begin{verbatim}
	context: Assistenza::inviaAssistenza(args)
	let t: task assistenza generata
	post: DatabaseTask.tasks->contains(t)

	context: Feedback::inviaFeedback(args)
	let t: task feedback generata
	post: DatabaseTask.tasks->contains(t)

	context: Riparazione::inviaRiparazione(args)
	let t: task riparazione generata
	post: DatabaseTask.tasks->contains(t)

\end{verbatim}
\section{ProfiloDipendente, ProfiloCliente, ProfiloManager}
\begin{figure}[H]
	\centering\includegraphics[width=1\textwidth]{images/OCL/OCL_dipendente.png}
	Diagramma delle classi con linguaggio OCL 
\end{figure}
\begin{figure}[H]
	\centering\includegraphics[width=1\textwidth]{images/OCL/OCL_cliente.png}
	Diagramma delle classi con linguaggio OCL 
\end{figure}
\begin{figure}[H]
	\centering\includegraphics[width=1\textwidth]{images/OCL/OCL_manager.png}
	Diagramma delle classi con linguaggio OCL 
\end{figure}
La data d'assunzione del dipendente dev'essere inferiore a quella odierna
\begin{verbatim}
	context: ProfiloDipendente inv: giornoAssunzione <= CURR_DATE
\end{verbatim}
ProfiloDipendente, ProfiloCliente, ProfiloManager hanno i PermissionLevel rispettivi
\begin{verbatim}
	context: ProfiloDipendente inv: permissionLevel = Dipendente
	context: ProfiloCliente inv: permissionLevel = Cliente
	context: ProfiloManager inv: permissionLevel = Manager
\end{verbatim}

\chapter{Diagramma delle classi con codice OCL}
\begin{figure}[H]
	\centering\includegraphics[width=1\textwidth]{images/Diagramma_delle_classi_OCL.png}
	Diagramma delle classi con linguaggio OCL 
\end{figure}
\end{document}
